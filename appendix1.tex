\chapter{Appendix}
%\label{cha:festschrift}


\section{Bayesian Linear Regression --- Different Conjugate Models and Their (In)Sensitivity to Prior-Data Conflict}
\label{sec:festschrift}

This section reproduces the work
``Bayesian Linear Regression --- Different Conjugate Models and Their (In)Sensitivity to Prior-Data Conflict'',
published as technical report no.~69 at the Department of Statistics of Ludwig-Maximilians-University Munich (LMU)
\parencite{Walter2009b}. This technical report is a substantially extended version of a contribution to
``Statistical Modelling and Regression Structures: Festschrift in Honour of Ludwig Fahrmeir'' \parencite{Walter2010a}.
As such, it is reproduced here verbatim, except for some minor rewording,
and the addition of some comments and footnotes linking this work to other parts of this thesis.
***Furthermore, the notation was changed slightly towards the one introduced in Section~\ref{sec:regularconjugates}
(most importantly, writing $\nn$ and $\yn$ for the canonical posterior parameters
instead of $n^{(1)}$ and $y^{(1)}$, respectively),
and citations were updated and changed to the style employed throughout this thesis.***

\subsubsection*{Abstract}

The paper is concerned with Bayesian analysis under
prior-data conflict, i.e.\ the situation when observed data are
rather unexpected under the prior (and the sample size is not large
enough to eliminate the influence of the prior). Two approaches for
Bayesian linear regression modeling based on conjugate priors are
considered in detail, namely the standard approach, described, e.g., in
\textcite{2013:fahrmeier-kneib-lang-marx}, and an alternative adoption of the
general construction procedure for exponential family sampling
models. We recognize that -- in contrast to some standard i.i.d.\
models like the scaled normal model and the Beta-Binomial /
Dirichlet-Multinomial model, where prior-data conflict is completely
ignored -- the models may show some reaction to prior-data conflict,
however in a rather unspecific way. Finally we briefly sketch the
extension to a corresponding imprecise probability model, where,
%by enlarging the underlying set of distributions and the imprecision of the resulting interval-valued probabilities,
by considering sets of prior distributions instead of a single prior,
prior-data conflict can be handled in a very appealing and intuitive way.


\subsection{Introduction}
\label{sec:festschrift-intro}

Regression analysis is a central tool in applied statistics that
aims to answer the omnipresent question how certain variables
(called covariates / confounders, regressors, stimulus or
independent variables, here denoted by $\x$) influence a certain
outcome (called response or dependent variable, here denoted by $z$).
Due to the complexity of real-life data situations, basic linear
regression models, where the expectation of the outcome $z_i$ simply equals the linear predictor $\x_i^\tra\bbeta$,
have been generalized in numerous ways, ranging from generalised linear
models (\cite{2001:fahrmeier-tutz}, see also \cite{1985:fahrmeir-kaufmann} for classical work on asymptotics)
for non-normal distributions of $z_i\mid \x_i$, or linear mixed models %(LMM, ***zitate)
allowing the inclusion of clustered observations, over semi- and nonparametric models
\parencite{2009:kauermann,2007:fahrmeir,2007:scheipl}, up to
generalised additive (mixed) models and structured
additive regression \parencite{2009:fahrmeir,2006:fahrmeir,2007:kneib}.

Estimation in such highly complex models may be based on
different estimation techniques such as (quasi-) likelihood, general
estimation equations (GEE) or Bayesian methods. Especially the
latter offer in some cases the only way to attain a reasonable
estimate of the model parameters, due to the possibility to include
some sort of prior knowledge about these parameters, for instance by
``borrowing strength'' \parencite[e.g.,][]{1996:higgins}.

The tractability of large scale models with their ever increasing complexity
of the underlying models and data sets should not obscure that
still many methodological issues are a matter of debate.
Since the early days of modern Bayesian inference one central issue has,
of course, been the potentially strong dependence of the inferences on the prior.
In particular in situations where data is scarce or unreliable, the
actual estimate obtained by Bayesian techniques may rely heavily on
the shape of prior knowledge, expressed as prior probability
distributions on the model parameters. %parameters of interest.
Recently, new arguments came into this debate by new methods
for detecting and investigating \emph{prior-data conflict} \parencite{2006:evans,2008:bousquet},
i.e. situations where  ``[\ldots] the observed data is surprising
in the light of the sampling model and the prior,
[so that] we must be at least suspicious about the validity
of inferences drawn.'' \parencite[p.~893]{2006:evans}%
\footnote{See also the discussion on prior-data conflict in
Sections~\ref{sec:motivation:pdc}, \ref{sec:pdc-sensitivity},
and in the paper reproduced in Section~\ref{sec:jstp}.}

The present contribution investigates the sensitivity of inferences on potential prior-data conflict:
What happens in detail to the posterior distribution, and the estimates derived from it, if prior knowledge and
what the data indicates are severely conflicting?
If the sample size $n$ is not sufficiently large to discard the
possibly erroneous prior knowledge and thus to rely on data only,
prior-data conflict should affect the inference and
should  -- intuitively and informally --  result in an increased degree
of uncertainty in posterior inference. Probably most statisticians would thus expect
a higher variance of the posterior distribution in situations of prior-data conflict.

However, this is by no means automatically the case,
in particular when adopting conjugate prior models,
which are often used when data are
scarce, where only strong prior beliefs allow for a reasonably
precise answer in inference. Two simple and prominent examples
of complete insensitivity to prior-data conflict are recalled in
Section~\ref{sec:iid}: i.i.d.\ inferences on the mean of a scaled normal distribution
and on the probability distribution of a categorical variable by the Dirichlet-Multinomial model.%
\footnote{See the description of these two models in Sections~\ref{sec:norm-norm} and \ref{sec:diri-multi}, respectively.}

Sections~\ref{sec:scp} and \ref{sec:cccp} extend
the question of (in)sensitivity to prior-data to regression models.
We confine attention to linear regression analysis with conjugate priors,
because -- contrary to the more advanced regression model classes --
the linear model still allows a fully analytical access, making it possible to understand
potential restrictions imposed by the model in detail.
We discuss and compare two different conjugate models:
\begin{enumerate}[(i)]
\item the standard conjugate prior (SCP, Section~\ref{sec:scp}) as described in \textcite{2013:fahrmeier-kneib-lang-marx} or,
in more detail, in \textcite{1994:ohagan}; and
\item a conjugate prior, called ``canonically constructed conjugate prior'' (CCCP, Section~\ref{sec:cccp}) in the following,
which is derived by a general method used to construct conjugate priors to sample distributions that
belong to a certain class of exponential families, described, e.g., in \textcite{2000:bernardosmith}.%
\footnote{This is the regular conjugate framework of Section~\ref{sec:regularconjugates}.}
\end{enumerate}
Whereas the former is the more general prior model, allowing for
a very flexible modeling of prior information (which might be welcome or not),
the latter allows only a strongly restricted covariance structure for $\bbeta$,
however offering a clearer insight in some aspects of the update process.

In a nutshell, the result is that both conjugate models do react
to prior-data conflict by an enlarged factor to the variance-covariance matrix
of the distribution on the regression coefficients $\bbeta$; however, this reaction is unspecific, as it affects the
variance and covariances of all components of $\bbeta$ in a uniform way
-- even if the conflict occurs only in one single component.

Probably such an unspecific reaction of the variance is the most a (classical)
Bayesian statistician can hope for, and traditional probability theory based on precise probabilities can offer.
Indeed, \textcite{1987:kyburg:ess} notes that
\begin{quotation}
\begin{small}
%``
[\ldots] there appears to be no way, within the theory, of
distinguishing between the cases in which there are good statistical grounds
for accepting a prior distribution, and cases in which the prior distribution
reflects merely ungrounded personal opinion.%''\\
\end{small}
\end{quotation}
and the same applies, in essence, to the posterior distribution.

A more sophisticated modeling would need a more elaborated concept
of imprecision than is actually provided by looking at the variance
(or other characteristics) of a (precise) probability distribution.
Indeed, recently the theory of imprecise probabilities \parencite{1991:walley,2001:weichselberger}
is gaining strong momentum.%
\footnote{See Section~\ref{sec:ip-intro} for a short exposition of the theoretical foundations
and motivations for the use of imprecise probability in statistical inference.}
It emerged as a general methodology to cope
with the multidimensional character of uncertainty, also reacting
to recent insights and developments in decision theory\footnote{See
\textcite{2005:hsu-bhatt} for a neuro science corroboration of the constitutive
difference of stochastic and non-stochastic aspects of uncertainty
in human decision making, in the tradition of Ellsberg's \parencite*{1961:ellsberg} seminal experiments.}
and artificial intelligence,\footnote{See, e.g., \textcite{1996:walley::expert}
for the use of imprecise probability methods in expert systems.}
where the exclusive role of probability
as a methodology for handling uncertainty has eloquently been rejected \cite{1999:klir}:
\begin{quotation}
\begin{small}
%``
For three hundred years [\ldots] uncertainty was conceived solely in
terms of probability theory. This seemingly unique connection
between uncertainty and probability is now challenged [\ldots by several
other] theories, which are demonstrably capable of characterizing
situations under uncertainty. [\ldots]

[\ldots] it has become clear that there are several distinct types of
uncertainty. That is, it was realized that uncertainty is a
multidimensional concept. [\ldots\ That] multidimensional nature of
uncertainty was obscured when uncertainty was conceived solely in
terms of probability theory, in which it is manifested by only one
of its dimensions.
\end{small}
\end{quotation}

Current applications include, among many other, risk analysis,
reliability modeling and decision theory, see \textcite{2009:ISIPTA},
\textcite{2011:ISIPTA} and \textcite{2009:CoolenSchrijner} for recent collections on the subject.%
\footnote{See also, e.g., the list of applications of the IDM given in Section~\ref{sec:idm-and-near-ignorance}.}
As a welcome byproduct, imprecise probability models also provide
a formal superstructure on models considered in robust Bayesian analysis \parencite{2000:rios}, and
frequentist robust statistic in the tradition of \textcite{1973:huberstrassen},
see also \textcite{2009:augustin-hable} for a review.

By considering \emph{sets} of distributions, and corresponding
interval-valued probabilities for events, imprecise probability models allow
to express the quality of the underlying knowledge in an elegant way.
The higher the ambiguity, the larger c.p.\ the sets.
The traditional concept of probability is contained as a special case,
appropriate if and only if there is perfect stochastic information.
This methodology allows also for a natural handling of prior-data conflict.
If prior and data are in conflict, the set of posterior distributions are enlarged,
and inferences become more cautious.%
\footnote{For more details on the topic of imprecision and \pdc, see Section~\ref{sec:jstp}.}

In Section~\ref{sec:discussion}, we briefly report that the CCCP model has a structure
that allows a direct extension to an imprecise probability model
along the lines of Quaeghebeur and de Cooman's \parencite*{2005:quaeghebeurcooman} imprecise probability models
for i.i.d.\ exponential family models. Extending the models further
by applying arguments from \textcite{Walter2009a}%
\footnote{Reproduced in Section~\ref{sec:jstp}.}
yields a powerful generalisation
of the linear regression model that is also capable of a component-specific reaction to prior-data conflict.

\subsection{Prior-data Conflict in the i.i.d.\ Case}
\label{sec:iid}


