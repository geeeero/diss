\chapter{Concluding Remarks}
\label{cha:concluding}


\section{***Summary / Discussion***}

write again about general model framework (Section~\ref{sec:generalmodel}),
inference properties etc.

example in Section


discussion of shapes as at the end of Section~\ref{sec:pdc-sensitivity} and \ref{sec:ibbm-resume}***

emphasise modeling opportunities, and realistic description and treatment
of model uncertainty as in \ref{sec:objections}




***comparison with other approaches here or in Section~\ref{sec:alternatives}?***

study by ***krautenbacher*** shows that elicitation of $\MZ$ can be difficult,
compares with \textcite{2005:whitcomb}, similar results****

more studies for comparison with **** to be done


\ref{sec:generalmodel}





\section{***Outlook***}

***avenues for further research***

***\textbf{first: model as is, use for\dots }

***model could be vey useful in statistical surveillance, cite IESS:surveillance?***,
as mentioned in footnote~\ref{foot:sequential}, page~\pageref{foot:sequential}),
write here some more: optimal stopping rules, outbreak detection is usually(?)
framed as a testing problem (will new data refute the current model?),
but can also be seen similar to \pdc\ (is new data in conflict with current (prior) model?).
it might be interesting to see if rectangular sets or other shapes are good for this.  

***use regression framework from Section~\ref{sec:cccp} in situations prone to \pdc.


***\textbf{then: extend the model, but keep GBR framework}

***extend to coarse or imprecise data, which is a very important field:
do not idealise also the data (we stopped idealising the prior already)!

from isipta11 concluding:
\begin{small}
For a deeper understanding of prior-data conflict, it may also be helpful
to extend our methods to coarse data, in an analogous way to \textcite{2007:utkinaugustin} and
\textcite{2009:Troffaes:Coolen}, and to look at other model classes of prior distributions, most
notably at contamination neighbourhoods. Of particular interest here may
be to combine both types of prior models, considering contamination
neighbourhoods of our exponential family based-models with sets of
parameters, as developed in the Neyman-Pearson setting by
\textcite[\S~5]{2002:augustin}.
\end{small}


***boatshape stuff: cater also for strong prior-data agreement***

as mentioned in ***isipta'11***, shape of $\MZ$ has a crucial influence on inferences,
and it might be very difficult to elicit a set \emph{shape} and ascertain*** the consequences of a certain shape.

although the updating of the canonical parameters \eqref{eq:canonicalupdate},
i.e., the weighted average update step for $\yz$, and the increment step for $\nz$,
seems very intuitive (and is central to the behaviour of the model,
see the list of inference properties in Section~\ref{sec:gbicp-properties-criteria}),
the shape change of $\MZ$ to $\MN$ through this updating and its effects on posterior inferences are difficult to grasp.

although update of a single coordinate $(\nz,\yz)$ is a simple shift,
this shift is different for different coordinates.

as shape change is so problematic for understanding,
is there a different parametrisation such that shape remains constant,
i.e., not only single coordinates shift, but also entire shapes
(same shift for all coordinates)?

yes, there is, see preliminary results by Mik Bickis (``personal communication'' if no citable work?***).
there, update step for $\nz$ is the same, and shift for parameter replacing $\yz$ is the same for all. 

describe some more?

rays of constant expectation (i.e., $\approx \yn$)

this parametrisation allows to tailor shapes for desired inference behaviour much easier.
indeed, GW and FC have thought of a shape that has \pdc\ sensitivity,
and gives `bonus precision' if prior and data agree especially well,
as mentioned a desirable property in Section~\ref{sec:insights}
and in the discussion of GBR in Section~\ref{sec:updating}

very appealing first results for Beta-Binomial model and Normal-Normal model,
hope for a joint publication of GW, FC and MB explaining this in detail.


\textbf{finally, general thoughts about updating, learning, etc}

elaborate thoughts in Section~\ref{sec:updating} again,
see also comments at end of Section~\ref{sec:6-gw-071216},

dual role of $\nz$ (Section~\ref{sec:ibbm-resume})

belief revision versus updating?

questions on (data) robustness of IP models, see isipta13 paper by Marco


