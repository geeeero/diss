\section{***Football example***}

For a match of team $A$ against team $B$, where $A$ is the home team,
we assume that for $g_A$, the number of goals scored by team A,
and $g_b$, the number of goals scored by team B, it holds that
\begin{align*}
g_A &\sim \po\Big(g_0 \frac{s_A}{s_B}\Big) \\
g_B &\sim \po\Big(g_0 \frac{s_B}{s_A}\Big) \,,
\end{align*}
where $g_0$ is the normal number of goals scored in a match,
and $s_A$ and $s_B$ are the strength of team $A$ and $B$, respectively.
Furthermore, $g_A$ and $g_B$ are assumed to be independent given the strength parameters $s_A$ and $s_B$.

The poisson assumption for the number of goals per team seems reasonable
from the study of league matches \parencite[see, e.g.,][p.~401 and Fig.~1]{2000:rue};
however, it is often assumed that matches with a very large number of goals
will too strongly influence the estimation,
and thus the number of goals by each team is truncated at a certain number
for the purpose of the model \parencite[e.g., 5, see][p.~402]{2000:rue}.
