\chapter{Generalized Bayesian Inference with Sets of Priors}

\begin{itemize}
\item Short introduction to IP (sets of priors, lower/upper prevision/probability)
%\item motivate imprecise priors with ress paper: \cite{Troffaes2013a} --- now in intro
\item \pdc\ (Evans \& Moshonov, Fuquene-Cook-Pericchi), examples (Festschrift paper: \cite{Walter2010a})
\item further motivations for IP (ITIP chapter \cite{itip-statinf}) 
\item Generalized Bayesian inference
\item Generalized Bayesian inference with sets of priors
 \begin{itemize}
 \item sets of priors, GBR, robust Bayes
 \item sets of conjugate priors in general
 \item parameter set shapes
 \end{itemize}
\item IDM
\item JSTP paper \cite{Walter2009a}
\item isipta11 paper \cite{Walter2011a}
\item boatshape?
\end{itemize}

After having seen a detailed example for Bayesian inference using sets of conjugate priors in Section~\ref{sec:commoncause},
in the main chapter of this thesis,
we will give now a general introduction to the methodology of Bayesian inference with sets of conjugate priors.
In Section ****, \\
Section **** 


\section{Imprecise or Interval Probability}
\label{sec:ip-intro}

This Section will give a condensed introduction to the main theoretic concepts
in interval or imprecise probability as needed for the topics discussed in this thesis.

While Section~\ref{sec:ip-general} follows mostly ***IESS -- Imprecise probability***,
Section~\ref{sec:ip-main} is based mainly on \textcite{1996:walley::expert} and \textcite{2000:walley::towards}.


\subsection{General Concept and Basic Interpretation}
\label{sec:ip-general}

The central idea of imprecise or interval probability \parencite{2011:IESS-ip, 1991:walley, 2001:weichselberger} is
to replace the usual, precise probability measure $P(A)$ for events $A$%
\footnote{Events of interest $A$ are taken to be subsets of the sample space $\Omega$,
forming a $\sigma$-algebra, which is non-empty and includes countable unions and intersections of subsets of $\Omega$.}
with a \emph{lower} and \emph{upper probability}, denoted by $\Pl(A)$ and $\Pu(A)$, respectively,
satisfying $0 \le \Pl(A) \le \Pu(A) \le 1$.
In this setting, a usual probability measure forms the extreme case $\Pl(A) = \Pu(A) = \p(A)$,
when there is enough information to determine the distribution on the sample space $\Omega$
in precise stochastic terms.
On the other extreme, when $\Pl(A) = 0$ and $\Pu(A) = 1$,
we have no information at all on the probability for $A$ to occur,
and intermediate cases $0 \le \Pl(A) < \Pu(A) \le 1$ represent
different degrees of knowledge on this probability.

Therefore, interval or imprecise probability adds another modeling dimension:
While usual, precise probability measures can be used to model phenomena when there is perfect stochastical information,
like, e.g., in a lottery where the number of winning tickets (and the total number of tickets) is precisely known,
imprecise probability measures can account for cases where there is uncertainty about the probabilities themselves,
just like in a lottery where the number of winning tickets is not exactly known.
Non-stochastic uncertainty about model features like probabilities is often called \emph{ambiguity},
forming a crucial part of the human decision process,
and there are studies suggesting that humans process ambiguity in a different way
as compared to pure probabilistic reasoning \parencite{2005:hsu-bhatt}.

In contrast to a probability measure $\p(A)$,
the set functions $\Pl(A)$ and $\Pu(A)$ do not adhere
to the additivity axiom (countable or finite additivity),
and thus are also known as \emph{non-additive probabilities}.
There is also a link to \emph{fuzzy measures}, which are also non-additive measures
\parencite[see, e.g.,][]{1997:denneberg}.

In general, $\Pl(A)$ may be understood as accounting for evidence certainly in favour of $A$,
and $\Pu(A)$ accounting for all evidence speaking not strictly against $A$.
The difference of $\Pu(A)$ and $\Pl(A)$ thus allows for inconclusive evidence
that may not speak unanimously in favor of or against $A$, respectively.
As evidence strictly against $A$ can be seen as evidence certainly in favour of $A^\com$,
the complement of $A$,
it is mostly assumed that $\Pl(A^\com) = 1 - \Pu(A)$,
and thus it suffices to determine either of $\Pl$ or $\Pu$,
the other one being defined through this relation.

Although strictly negated by advocates of Bayesian methods \parencite[e.g., by][]{1987:lindley},
the need to go beyond usual probability measures has been recognized for a long time,%
\footnote{\textcite{2009:hampel} and \textcite[Section~1]{2001:weichselberger} give a historical overview on the development of
ideas related to non-additive measures and interval probability.}
and in recent times most prominently by scientists involved in the development of expert systems.


main monographs: Weichselberger (axiomatic), Walley (subjective Bayesian)

\subsection{Main Formulations}
\label{sec:ip-main}

lower and upper probabilities or expectations,
link to sets of probability distributions,
sets of desirable gambles as mathematical formulation


\subsection{Related Concepts}

many links to concepts that aim to complement probability theory as tool for handling uncertainty,
like possibilistic reasoning, fuzzy probabilities, etc.


\section{***Motivations for IP}

\subsection{\pdc }

\subsection{Weakly Informative Priors}


\subsection{Other Motives}

%for later in IDM/IBBM discussion:\\
%we use Walley's \cite[\S 7.7.3, p.~395]{1991:walley} $(s,\vec{t})$ notation for the hyperparameters.
