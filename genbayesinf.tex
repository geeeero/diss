\chapter{Generalized Bayesian Inference with Sets of Priors}

\begin{itemize}
\item Short introduction to IP (sets of priors, lower/upper prevision/probability)
%\item motivate imprecise priors with ress paper: \cite{Troffaes2013a} --- now in intro
\item \pdc\ (Evans \& Moshonov, Fuquene-Cook-Pericchi), examples (Festschrift paper: \cite{Walter2010a})
\item further motivations for IP (ITIP chapter \cite{itip-statinf}) 
\item Generalized Bayesian inference
\item Generalized Bayesian inference with sets of priors
 \begin{itemize}
 \item sets of priors, GBR, robust Bayes
 \item sets of conjugate priors in general
 \item parameter set shapes
 \end{itemize}
\item IDM
\item JSTP paper \cite{Walter2009a}
\item isipta11 paper \cite{Walter2011a}
\item boatshape?
\end{itemize}

After having seen a detailed example for Bayesian inference using sets of conjugate priors in Section~\ref{sec:commoncause},
in the main chapter of this thesis,
we will give now a general introduction to the methodology of Bayesian inference with sets of conjugate priors.
In Section ****, \\
Section **** 


\section{Imprecise or Interval Probability Calculus}
\label{sec:ip-intro}

not single number, but interval: old idea\\
adding another modeling dimension

main monographs: Weichselberger (axiomatic), Walley (subjective Bayesian)

lower and upper probabilities or expectations,
link to sets of probability distributions,
sets of desirable gambles as mathematical formulation

coherent lower probabilities etc.


many links to concepts that aim to complement probability theory as tool for handling uncertainty,
like possibilistic reasoning, fuzzy probabilities, etc.


\section{***Motivations for IP}

\subsection{\pdc }

\subsection{Weakly Informative Priors}


\subsection{Other Motives}

%for later in IDM/IBBM discussion:\\
%we use Walley's \cite[\S 7.7.3, p.~395]{1991:walley} $(s,\vec{t})$ notation for the hyperparameters.
