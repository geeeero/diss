\chapter{Introduction}


\begin{itemize}
\item overview, which paper used where\ldots
\item statistical inference ??? (\textbf{StatInf chapter}, \cite{itip-statinf})
\item Bayesian inference, conjugate priors, etc. (\textbf{TechRep Dirichlet}, \cite{Walter2012b})
\item motivate priors etc.\ with ress paper: \cite{Walter2013a}
\item short part on decision theory, updating rules?
\item \pdc\ (Evans \& Moshonov), examples (\textbf{Festschrift paper}, \cite{Walter2010a})
\item other motivations for IP?
\end{itemize}

\section{Overview}

***structure, which papers where***

Sec.~\ref{sec:stat-inference}: parts of \cite{itip-statinf}

Sec.~\ref{sec:bayes-inference}: parts of \cite{itip-statinf}

\section{Statistical Inference}
\label{sec:stat-inference}

%This thesis being concerned with generalized Bayesian Inference

In this Section, we will give a short introduction to statistical inference
and models that are used to describe random samples.
%These introductory sections may also serve as a guide to the notation utilized.

Statistical inference is about learning from data.
It is basically concerned with inductive reasoning,
i.e.\ establishing a general rule from observations.
As is long known as the problem of induction \cite{1739:hume},
it is impossible to justify inductive reasoning by pure reason,
and therefore one cannot infer general statements (laws) with absolute truth from single observations.
%***statistical models are built for a certain purpose. (citexxx George P. Box, Kaplan?)\\
%***models try to mirror certain aspects of reality: those deemed relevant to answer %the question(s) at hand etc.
The statistical remedy for this inevitable and fundamental dilemma of any type of inductive reasoning is
(postulated, maybe virtual) \emph{randomness} of the sampling process that generates the data.
If, and only if, the sample is, or can be understood as, drawn randomly,
probability theory allows to quantify the error of statistical propositions concluded from the sample.

Specifically, to model the randomness, a \emph{statistical model} is formulated.
It is a tuple $(\mathcal{X}, \mathcal{Q})$, consisting of the \emph{sample space} $\mathcal{X}$,
i.e.\ the domain of the random quantity $X$ under consideration,
and a set $\mathcal{Q}$ of probability distributions,%
\footnote{Most models of statistical inference rely on $\sigma$-additive probability distributions.
Therefore, technically, in addition an appropriate ($\sigma$-)field $\sigma(\mathcal{X})$,
describing the domain of the underlying probability measure, has to be specified.
In most applications there are straightforward canonical choices for $\sigma(\mathcal{X})$,
and thus $\sigma$-fields are not explicitly discussed here.}
collecting all probability distributions that are judged to be potential candidates for the distribution of $X$.
In this setting $\mathcal{Q}$ is called \emph{sampling model} and every element $p\in \mathcal{Q}$ \emph{(potential) sampling distribution}.
The inferential task is to learn the true element $p^* \in \mathcal{Q}$ from multiple observations of the random process producing $X$.

In this thesis, generally, so-called \emph{parameteric models} are considered,
where $\mathcal{Q}$ is pa\-ra\-me\-trized by a parameter $\vartheta$ of finite dimension,
assuming values in the so-called \emph{parameter space} $\Theta$, $\Theta \subseteq \reals^q, \ q < \infty$,
i.e.\ ${\cal Q}=\left(p_\vartheta\right)_{\vartheta \in \Theta}$.
Here, the different sampling distributions $p_\vartheta$ are implicitly understood as belonging to a specific class of distributions
(e.g. normal distributions, see the example in ****),
the basic type of which is assumed to be known completely,
and only some characteristics $\vartheta$ (e.g.\ the mean) of the distributions are unknown.

Throughout, we will assume (as is the case for all common applications)
that the underlying candidate distributions $p_\vartheta$ of the random quantity $X$
are either discrete or absolutely continuous with respect to the Lebesgue measure
(see, e.g. \cite[pp.~32f, 38]{1993:karr} for some technical details) for every $\vartheta \in \Theta$.
Then it is convenient to express every $p_\vartheta$ in the discrete case by its \emph{mass function} $f_\vartheta$,
with $f_\vartheta(x):=p_\vartheta(X=x),\forall x \in \mathcal{X}$,
and in the continuous case by its \emph{probability density function} (pdf) $f_\vartheta$,
where $f_\vartheta$ is such that $p_\vartheta(X \in [a,b]) = \int_a^b f_\vartheta(x) \dd x$.

An \emph{i.i.d.\ sample of size} $n$ (where \emph{i.i.d.}\ abbreviates independent, identically distributed)
\emph{based on the parametric statistical model} %
$(\mathcal{X}, (p_\vartheta)_{\vartheta \in \Theta})$ is a vector
\[
{\X}=(X_1, \ldots, X_n)^T %\label{120722-1}
\]
of independent random quantities $X_i$ with the same distribution $p_\vartheta$.
Then $\X$ is defined on $\mathcal{X}^n$ with probability distribution $p_\vartheta^{\otimes n}$
as the $n$-dimensional product measure describing the independent observations.
For Bayesian approaches as discussed here,
independence is often replaced by exchangeability (see, e.g., \cite[\S 4.2]{2000:bernardosmith}).
$p_\vartheta^{\otimes n}$ thus has the probability mass or density function
\[
\prod_{i=1}^{n} f_\vartheta(x_i) =: f_\vartheta(x_1, \ldots, x_n).%\label{120710-2}
\]
The term \emph{sample} is then also used for the concretely observed value(s) $\x=(x_1, \ldots, x_n)^T$.%
\footnote{Throughout, random quantities are denoted by capital letters, their values, or realizations, by small letters.}



\section{Statistical Inference with the Bayesian Paradigm}
\label{sec:bayes-inference}

As the inference models discussed in this thesis are all based on the Bayesian approach to statistical inference,
we will now give a short introduction to the basic principles of Bayesian inference.

The Bayesian approach allows (possibly subjective) knowledge on the parameter $\vartheta$ to be expressed by a probability distribution on%
\footnote{Again we implicitly assume that $\Theta$ is complemented by an appropriate $\sigma$-field $\sigma(\Theta)$.}
$\Theta$, with probability mass or density function $p(\vartheta)$ called \emph{prior distribution}.
Interpreting the elements $f_\vartheta(x)$ of the sampling model as conditional distributions of the sample given the parameter,
denoted by $f(\mbf{x}|\vartheta)$ and called \emph{likelihood},
turns the problem of statistical inference into a problem of probabilistic deduction,
where the \emph{posterior distribution}, i.e.\ the distribution of the parameter given the sample data,
can be calculated by Bayes' Rule.%
\footnote{Donald Gillies \cite{1987:gillies, 2000:gillies} argues that Bayes' Theorem was actually developed
in order to confront the problem of induction as posed by Hume \cite{1739:hume}.}
Thus, in the light of the sample $\mbf{x}= (x_1, \ldots, x_n)$ the prior distribution is updated by Bayes' Rule
to obtain the posterior distribution with density or mass function
\begin{align}
\label{eq:bayesrule}
p(\vartheta|\mbf{x}) \propto f(\mbf{x}|\vartheta) \cdot p(\vartheta)\,.
\end{align}
The posterior distribution is understood as comprising all the information from the sample and the prior knowledge.
It therefore underlies all further inferences on the parameter $\vartheta$,
like point estimators, interval estimators,
or the \emph{posterior predictive distribution},
which is the distribution of further observations based on $p(\vartheta|\mbf{x})$ (see below).



