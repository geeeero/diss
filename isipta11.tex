\section{On Prior-Data Conflict in Predictive Bernoulli Inferences}
\label{sec:isipta11}

%***with Frank's weighting model!***
***abstract***

By its capability to deal with the multidimensional nature of
uncertainty, imprecise probability provides a powerful methodology
to sensibly handle \pdc\ in Bayesian inference. When there is
strong conflict between sample observations and prior knowledge, the posterior model should be more imprecise
than in the situation of mutual agreement or compatibility. Focusing
presentation on the prototypical example of Bernoulli
trials, we discuss the ability of different approaches to deal with \pdc.

We study a generalised Bayesian setting, including Walley's Imprecise Beta-Binomial model
and his extension to handle prior data conflict (called pdc-IBBM here).
We investigate alternative shapes of prior parameter sets, chosen in a way that shows improved
behaviour in the case of \pdc\ and their influence on the posterior predictive distribution.
Thereafter we present a new approach, consisting of an
imprecise weighting of two originally separate inferences, one of which is based on an informative
imprecise prior, whereas the other one is based on an uninformative imprecise prior. This approach
deals with \pdc\ in a fascinating way.

***end of abstract***

\medskip

***from introduction***

In this paper a deeper investigation of the issue of \pdc\ is undertaken,
focusing on the prototypic special case of predictive inference in Bernoulli trials:%
\footnote{See also the discussion of the Beta-Binomial model in Section~\ref{sec:beta-binom},
and the imprecise Dirichlet-Multinomial model discussed in several examples in Section~\ref{sec:jstp}.}
We are interested in the posterior predictive
probability for the event that a future Bernoulli random quantity
will have the value $1$, also called a `success'. This event is not
explicitly included in the notation, i.e.\ we simply denote its lower
and upper probabilities by $\Pl$ and $\Pu$, respectively. This future Bernoulli random
quantity is assumed to be exchangeable with the Bernoulli random
quantities whose observations are summarised in the data, consisting
of the number $n$ of observations and the number $s$ of these that are
successes. In our analysis of this model, we
will often consider $s$ as a a real-valued observation in $[0,n]$,
keeping in mind that in reality it can only take on
integer values, but the continuous representation is convenient for
our discussions, in particular in our predictive probability plots (PPP),
where for given $n$, $\Pl$ and $\Pu$ are discussed as functions of $s$.

\medskip

Section~\ref{sec:ibbm-framework} describes a general framework for
generalized Bayesian inference in this setting. The method presented in \textcite[\S 5.4.3]{1991:walley},
called `pdc-IBBM' in this paper, is considered in detail in Section~\ref{sec:ibbm-walley}
and we show that its reaction to \pdc\ can be improved
by suitable modifications of the underlying imprecise
priors. A basic proposal along these lines is discussed in
Section~\ref{sec:othershapes} with further alternatives
sketched in Section~\ref{sec:ibbm-resume}.
Section~\ref{sec:weightedinf} addresses the problem of \pdc\ from a
completely different angle. There, we combine two originally separate
inferences, one based on an informative imprecise prior and one
on an uninformative imprecise prior, by an imprecise weighting
scheme. The ***paper*** concludes with a brief comparison of the different
approaches.






\section{***boatshape stuff in outlook?***}


