\section{***Generalised iLUCK-models (JSTP paper)***}
\label{sec:jstp}

****short intro from intro and abstract, examples already in Chapter~\ref{cha:intro}


\medskip

***The section is structured as follows:
In Section~\ref{070517-sec2-1} we provide the formal background
by distinguishing a wide class of classical, precise probability models
where Bayesian inference has a particular form, being directly
suitable to a generalization to imprecise probabilities by varying
the linearly updated main parameter (Section~\ref{sec:3}).
We illustrate these models and then show in Section~\ref{sec:4-gw-071216}
how these models can be extended to deal with prior-data conflict in a more sensible way
by additionally varying another parameter.
Section~\ref{sec:illu} illustrates our procedure in two important special cases:
inferences on the mean of a normal distribution and in the IDM.***


\subsection{Traditional Bayesian Inference and LUCK-models}\label{070517-sec2-1}

%\section{Bayesian Inference and \textsc{luck}-models}\label{070516-sec2}
%\subsection{Classical Bayesian Inference and \textsc{luck}-models}\label{070517-sec2-1}
In most cases, traditional Bayesian inference is based on so-called
conjugate priors (related to a specific likelihood). These
distributions have the convenient property that the posterior
resulting from~\eqref{eq:bayesrule} belongs to the same class of parametric
distributions as the prior. The posterior thus remains easily
tractable, and updating can be described in terms of parameters
only. For describing the imprecise probability model used later on more easily, we want to
%For the intended application presented later on, it is quite convenient to
distinguish certain standard situations (called
\emph{models with `Linearly Updated Conjugate prior Knowledge'
(LUCK)} here) of Bayesian updating with classical (traditional, precise)
probabilities, where prior and posterior fit nicely together in the sense that
(i) they belong to the same class of parametric distributions, and, in addition,
(ii) the updating of one parameter ($\yz$ below) of the prior is
linear given a second parameter ($\nz$).
%\begin{enumerate}%
%\vspace*{-1.2ex}%
%\item[i)] they belong to the same class of parametric distributions,
%and, in addition,\vspace*{-1.2ex}
%\item[ii)] the updating of one parameter ($y^{(0)}$ below) of the prior is
%linear given a second parameter ($n^{(0)}$).
%\end{enumerate}
%\vspace*{-1.2ex}%
More precisely, we return to the following definition,
originally introduced in \textcite{Walter2007a}:%\vspace*{-0.5ex}
\footnote{Compare to the model presented in Section~\ref{sec:regularconjugates}.
Here, the likelihood $f(\x\mid\vartheta)$ may have any form,
given that the update step from prior to posterior distribution
adheres to Equations~\eqref{070305-2} -- \eqref{070305-4}.
The model discussed in Section~\ref{sec:regularconjugates} is a special case,
as Equations~\eqref{070305-2} and \eqref{070305-4} are the same as
Equations~\eqref{eq:canonicalprior} and \eqref{eq:canonicalupdate}, respectively.
See Example~\ref{ex:jstp-1} below.}

\begin{definition}[LUCK-models]\label{070503-defin1}
Consider traditional Bayesian inference on a parameter $\vartheta$
based on a sample $\x$ as described in Section~\ref{sec:bayes-inference} (Equation~\eqref{eq:bayesrule}),
and let the prior $p(\vartheta)$ be characterized by the (vectorial) parameter
$\vartheta\uz$. The pair $\big(p(\vartheta),
p(\vartheta\mid\x)\big)$ is said to constitute a
\emph{LUCK-model of size $n\in \naturals$ with respect to the
likelihood $f(\x\mid\vartheta)$ in the natural parameter $\psi$ with
prior parameters $\nz \in \posreals$ and $\yz$ and sample
statistic $\tau(\x)$} iff $p(\vartheta)$ and $p(\vartheta\mid\x)$
can be rewritten in the following way:%
%\footnote{$\langle a,b \rangle$ denotes the scalar product of $a$ and $b$.}%\vspace*{-1.2ex}
\begin{equation}\label{070305-2}
p(\vartheta) \propto \exp\big\{ \nz \big[\langle \psi, \yz \rangle - \mbf{b}(\psi)\big]\big\} %\vspace*{-1.5ex} %\label{equ:QdCpriori}
\end{equation}%
and %\vspace*{-1.0ex}
\begin{equation}\label{070305-3}
p(\vartheta\mid\x) \propto \exp\big\{ \nn \big[\langle \psi, \yn \rangle - \mbf{b}(\psi)\big]\big\}\,,%\vspace*{-1.5ex} %\label{equ:QdCpriori}
\end{equation}
with
\begin{align}\label{070305-4}
%n^{(1)} &= n^{(0)}+n & &\mbox{and}  & y^{(1)} &= \frac{n^{(0)}y^{(0)}+\tau(x)}{n^{(0)}+n}\,,\vspace*{-3.5ex}  %\label{equ:QdCupdate}
\yn &= \frac{\nz}{\nz + n} \cdot \yz + \frac{n}{\nz + n} \cdot \frac{\tau(\x)}{n}\,, &
\nn &= \nz + n\,,
\end{align}%\vspace*{-1.0ex}%
where $\vartheta$ is transformed to $\psi$ and $\mbf{b}(\psi)$,
and $\vartheta\uz$ to $\nz$ and $\yz$.
\end{definition}%\vspace*{-0.5ex}%\vspace*{-1.9ex}%
%
$\yz$ and $\yn$ can be seen as the parameter describing
the main characteristics of the prior and the posterior,
respectively, and so later on, $\yz$ and $\yn$ will be
called \emph{main prior} and \emph{main posterior parameter}.
In the models considered here, $\yz$ can also be understood as a prior
guess for the random quantity $\ttau(\x) := \tau(\x)/n$
summarizing the sample. According to the left
part of \eqref{070305-4} these two different sources of information
are linearly combined to obtain the main posterior parameter:
%\vspace*{-1.3ex}
\begin{equation}\label{071214-2}
\yn = \frac{\nz}{\nz + n} \cdot \yz + \frac{n}{\nz + n} \cdot \ttau(\x)\,. %\vspace*{-1.0ex}%\frac{1}{q}\tau(w).
\end{equation}
This relation also equips $\nz$ with a vivid interpretation as ``prior strength''
or as ``pseudocounts'',
%which will become clearer in Section~\ref{070514-secImpPriorLUCK-QdC}.
reflecting the weight one gives to the prior with respect to the sample.
So, $\nz$ can be
interpreted as the size of an imaginary sample that corresponds to
the trust on the prior information in the same way as the sample
size of a real sample corresponds to the trust in conclusions based
on such a real sample. %
%
%\altrev{This interpretation is supported by the left part of
%(\ref{070305-4}): If in consecutive Bayes learning the posterior is
%used as the new prior, then the new main prior parameter $y\uo$ gets
%prior strength $n\uo = n\uz + n$.}%\gwc[mehr zu pseudocounts in BernardoSmith?]{}
%\gwc[hier schon was zu vektoriellen Größen?]{}
%\sidenote{Absatz zu $\vartheta\uz$ vectorial auskommentiert}
%where consistently throughout the paper vectors are evaluated component
%by component, here possibly leading to a vector-valued quantity.
%
%If $\vartheta\uz$ is vectorial, it is possible that $y\uz$ and
%$\tilde{\tau}(w)$ are vectors as well. In this case all functionals
%of vectors in this paper \thccc{are} evaluated component by
%component, leading to a vector-valued quantity.
%
As a preparation for the generalizations considered later, let us
turn to some characteristic examples, also illustrating the
interpretations of $\yz$ and $\nz$.

\begin{example}[Bayesian Inference in Exponential Families]
\label{ex:jstp-1}
%\noident\textbf{Example 1: Bayesian Inference in Exponential Families.}
In the case of independently and identically
distributed (i.i.d.) observations $\x = (x_1,\ldots, x_n)$ from
\emph{regular canonical exponential families} \parencite[p.~202 and p.~272f]{2000:bernardosmith},
a general result (see, e.g., ibid., Proposition~5.4) is available on how
to construct conjugate priors. A prior obtained by this method then
constitutes a LUCK-model of size $n$ (the sample size)
with the sample statistic of the whole sample being the sum of
statistics for each sample element (which can be concluded from the
canonical form of the likelihood), so $\tau(\x) = \sum_{i=1}^n \tau^*(x_i)$,
and $\ttau(\x) = \frac{1}{n}\sum_{i=1}^n \tau^*(x_i)$.%
\footnote{\textcite{2005:quaeghebeurcooman} consider this special case of LUCK-models in
their seminal work motivating the generalisations presented here.}
This is effectively the framework of Bayesian inference with regular conjugate priors
as presented in Section~\ref{sec:regularconjugates}.
To perceive the generality of this result, recall that many of the
sample models most often used in practice form an exponential family,
as shown earlier in this thesis
for the Binomial distribution (Section~\ref{sec:beta-binom}),
the Normal or Gaussian distribution (Section~\ref{sec:norm-norm}),
and the Multinomial distribution (Section~\ref{sec:diri-multi}).
%For samples from a scaled normal and from a
%multinomial distribution, the derived priors are presented explicitly here:
\end{example}

\iffalse
\noindent\textbf{Example 1a: Samples from a scaled Normal $\mbox{N}(\mu,1)$.}
The conjugate distribution to an i.i.d.-sample of size $n$ from a
scaled normal distribution with mean $\mu$, denoted by $\mbox{N}(\mu,1)$,
as constructed by this method is a normal distribution with
mean $y\uz$ and variance $\frac{1}{n\uz}$. Here, we have $\psi = \mu$,
$\mbf{b}(\psi) = \frac{\mu^2}{2}$, and $\tau(x_i) = x_i$,
thus $\tilde{\tau}(x) = \bar{x}$, and\vspace*{-1.5ex}
\begin{eqnarray*}
p(\mu) &\propto& \exp \left\{ -\frac{n\uz}{2} \left(\mu - y\uz\right)^2 \right\}
        \propto  \exp \left\{ n\uz \left[ \mu \cdot y\uz - \frac{\mu^2}{2} \right] \right\}\,.
%      \,\hat{=}\, \exp \big\{ n\uz \big[ \langle \psi, y\uz \rangle - \mbf{b}(\psi) \big] \big\}\,.
\end{eqnarray*}
%\begin{eqnarray*}
%\prod_{i=1}^n p(w_i\,|\,\mu) &=& \prod_{i=1}^n \frac{1}{\sqrt{2 \pi }} \exp \{ -\frac{1}{2}(w_i - \mu)^2 \} \\
%                             &=& \frac{1}{(2 \pi)^\frac{n}{2}} \exp \{ -\frac{1}{2} \sum_{i=1}^n (w_i - \mu)^2 \}
%\end{eqnarray*}
%\medskip
\textbf{Example 1b: Samples from a Multinomial $\mbox{M}(\btheta)$.}
Given a sample of size $n$ from a multinomial distribution with
probabilities $\theta_j$ for categories $j= 1,\ldots,k$, subsumed in
the vectorial parameter $\btheta$ (with $\sum_{j=1}^k \theta_j =1$),
the conjugate prior on $\btheta$ constructed along the general
method \thcc{can be shown to be} the Dirichlet distribution
$\mbox{Dir}(\bs{\alpha})$. Written in terms of
Walley's \citeyearpar{Walley-IDM} reparameterization
$\alpha_j = s\cdot t_j$ as used later on, it holds that $n\uz = s$
and $y\uz = \mbf{t}=(t_1,\ldots,t_k)^\tra$. Note in
particular that here the components of $y^{(0)}$ have a direct
interpretation as class probabilities. This so-called
Dirichlet-Multinomial model provides the basis of the Imprecise
Dirichlet Model (IDM), introduced by \cite{Walley-IDM}, that is
considered in the continuation of this example in Section~\ref{sec:3}.%
\fi

\begin{example}[Bayesian Inference in Linear Regression]
%\noindent\textbf{Example 2: Bayesian Inference in Linear Regression.}
As shown by \textcite{Walter2006a,Walter2007a,Walter2007b},
the importance of LUCK-models is not limited to the i.i.d.\ case
but also provides a formal superstructure containing in particular
the practically important case of linear regression
models, modelling the (linear) influence of certain variables (called covariates,
confounders, regressors, stimulus or independent variables) on a
certain outcome (also called response or dependent variable).%
\footnote{See Chapter~\ref{cha:festschrift} (Appendix~1) for alternative models
that adhere to the regular conjugate framework of Section~\ref{sec:regularconjugates}.}
\end{example}


\subsection{Imprecise Priors for Inference in LUCK-models}\label{sec:3}%\label{070514-secImpPriorLUCK-QdC}

Our definition of LUCK-models was inspired by the work of
\textcite{2005:quaeghebeurcooman}, who develop a general approach for inference with
imprecise priors, which was proven in \textcite{Walter2007a} to be
generalisable to arbitrary LUCK-models. Quaeghebeur
and de Cooman's seminal idea was that the seemingly strange
parameterisation in terms of $\yz$ and $\nz$ in
\eqref{070305-2} and \eqref{070305-3} is perfectly suitable to be
generalised to credal sets of priors. The crucial point is that
$\yz$, the main prior parameter%
%\footnote{Example 1a: the prior mean for $\mu$;
%Example 1b: the vector of prior expected values for probabilities $\btheta$}
\footnote{In the Normal-Normal model: the prior mean for $\mu$;
in the Dirichlet-Multinomial model: the vector of prior expected values
for the category probabilities $\btheta$.}
is updated \emph{linearly}, when
the prior strength $\nz$ is taken as fixed. This makes an
easily tractable imprecise calculus possible: When sets of priors
are defined via sets of main parameters $\yz$, and these sets of
$\yz$ are defined by lower and upper bounds, the lower and upper
bounds of the sets of posterior main parameters $\yn$ can be
obtained directly from \eqref{070305-4}.

\subsubsection{\ymodel s}\label{070514-secImpPriorLUCK-QdC}
%\kom{erster Satz auch eher in Introduction?\\} Several powerful
%approaches have been proposed to overcome the ``dogma of ideal
%precision'' (Walley) underlying classical Bayesian inference (cf.,
%in particular, \mycite{PericchiWalley, Coolen-Artikel,
%QdC[2], QdC}; see also Section~\ref{070306-ConRem}).\\

In constructing a conjugate prior to a given likelihood,
Quaeghebeur and de Cooman strictly rely on the method described in Section~\ref{sec:regularconjugates}
(Example~\ref{ex:jstp-1}), but their technique to construct an \emph{imprecise}
conjugate prior, i.e.\ a set of priors, by considering a set of
$\yz$, does not depend on this derivation, but rather on the
linearity of the updating of values of $\yz$ given $\nz$ when
the set of posterior distributions is calculated. As the
LUCK-models capture exactly this property, it is possible
to construct \emph{imprecise} conjugate priors for arbitrary
LUCK-models according to Quaeghebeur and de Cooman's
technique.\footnote{See \textcite{Walter2006a,Walter2007a,Walter2007b}
for an application of this idea to obtain
linear regression models with imprecise prior distributions.} We
give a general exposition of this model class, illustrate it by
continuing the previous examples, and elaborate their serious
deficiency with respect to the handling of prior-data conflict,
which then will be overcome in Section~\ref{sec:4-gw-071216}.

Due to the linearity of the updating for fixed $\nz$, minimisation and maximisation problems on the set of
posteriors can be reduced to minimisation and maximisation problems
on the set of priors when the parameter $\yn$ (or a linear function of it) is the quantity of interest. This
very same update procedure is used in the Imprecise Dirichlet Model
(IDM), which is based on the Dirichlet-Multinomial model as
presented in Section~\ref{sec:diri-multi}.
Just as Walley required $s$ to be fixed in the IDM, Quaeghebeur
and de Cooman consider sets of $\yz$ but only a single value for
the other prior parameter, denoted by $\nz$ here. As we will show
in detail in Section~\ref{section:fixednschlecht}, the resulting models
necessarily ignore prior-data conflict. However, as a
preparation for our generalisation presented in
Section~\ref{sec:4-gw-071216}, we want to present the model with
varying $\yz$ but fixed $\nz$, which will be called \ymodel\ (for
\emph{imprecise} LUCK-model), in more detail:
%
%
%
\begin{definition}[iLUCK-models]\label{071219-def2}
Consider the situation of Definition~\ref{070503-defin1},
and a set of LUCK-models $\big(p(\vartheta),p(\vartheta\mid\x)\big)$
(with respect to the likelihood $f(\x\mid\vartheta)$ in the natural
parameter $\psi$ with prior parameters $\nz \in \posreals$ and
$\yz$ and sample statistic $\tau(\x)$), produced by $\yz$
varying in some set $\YZ \subset \Y$, where the
parameter space $\Y$ is taken as the convex hull (without
the boundary) of the range of $\tau(\x)$. Let furthermore the credal
sets ${\cal M}$ and ${\cal M}_{|\x}$ consist of all convex mixtures
obtained by this variation of $p(\vartheta)$ and $p(\vartheta\mid\x)$.
Then $\big({\cal M}, {\cal M}_{|\x}\big)$ is called the corresponding
\emph{imprecise LUCK-model (\ymodel)} based on $\YZ$ and $\nz$.
\end{definition}

\begin{remark}
Note that if ${\cal M}$ is used as an imprecise prior, by
construction, ${\cal M}_{|\x}$ is the corresponding imprecise
posterior. Although the imprecise prior contains not only
the parametric distributions, but also arbitrary convex mixtures of them,
it is nevertheless easy to obtain the imprecise posterior: Since it
is sufficient to update the extreme points and the updating process
is linear in $\YZ$, the imprecise posterior ${\cal M}_{|\x}$ is simply
obtained as the set of all convex mixtures of posteriors
$p(\vartheta\mid\x)$ arising from \eqref{070305-3} by varying
$\yn$ in $\YN$,
where%\vspace*{-0.7ex}%
\begin{alignat}{3}
\YN &= \left\{ \left.\frac{\nz\yz
                  + \tau(\x)}{\nz+n} \;\right|\; \yz \in \YZ \right\}
%                    \subset {\cal Y}\,. %\label{equ:QdCYk}
               &\,=\, \frac{\nz}{\nz + n} \cdot \YZ
                + \frac{n}{\nz + n} \cdot \ttau(\x)\,.
\label{equ:shifttrans}%\vspace*{-2.5ex}%
\end{alignat}%\vspace*{-3.5ex}%
\end{remark}
%
%
In generalisation of (\ref{071214-2}), $\YN$ can
actually be seen as a shifted and rescaled version of $\YZ$,
which allows us to keep the vivid interpretation of  $\nz$ as
``prior strength'' or as ``pseudocounts'', as it plays again the same role
for the prior as $n$ for the sample.\footnote{$\YZ$ must be
bounded, as for any $\yzu = \infty$, it holds that
$\ynu = \infty$ as well. For the IDM, introducing explicit
bounds is not necessary, as the parameter space $\Y$ itself
is already bounded, being the unit simplex.}
%
%
%
\begin{remark}\label{071221-remark}
In iLUCK-models, the ``magnitude'' of $\YZ$ and $\YN$ naturally reflects
the imprecision in the prior and the posterior, respectively.
Consequently, we will define\footnote{If the main parameter is
multidimensional (denoted by $\byi, i=0,n$), then, throughout the paper, the infimum, the
supremum, and the measure ${\rm MPI}\ui$ and related quantities,
are to be understood as defined component by component. Natural
choices for real-valued measures derived from vector-valued ${\rm MPI}\ui$
would be to consider appropriate norms.}, with%\vspace*{-0.8ex}
\begin{equation}\label{071214-3}
\yil := \inf\!\left\{\left.\! \yi\,\right|\, \yi\!\in \YI \!\right\}\;%\;
%\ul{y}^{(i)}:= \inf\!\left\{\left.\! y^{(i)}\,\right|\, y^{(i)}\!\in {\cal Y}^{(i)} \!\right\}\;%\;
\text{and}\;\;
\yiu := \sup\!\left\{\left.\! \yi\,\right|\, \yi\!\in \YI \!\right\}\!, \; i=0,n\,,%\vspace*{-0.7ex}
%\ol{y}^{(i)}:= \sup\!\left\{\left.\! y^{(i)}\,\right|\, y^{(i)}\!\in {\cal Y}^{(i)} \!\right\}\!, \; i=0,1\,,\vspace*{-0.7ex}
\end{equation}
the main parameter prior imprecision and main parameter
posterior imprecision ${\rm MPI}\uz$ and ${\rm MPI}\un$ by%\vspace*{-0.3ex}
\begin{equation}\label{071214-4}
{\rm MPI}\ui := \yiu - \yil,\quad i=0,\,n.%\vspace*{-0.3ex}%
\end{equation}
%%MPI$^{(0)}$ and MPI$^{(1)}$ are called .
A natural tool to summarize basic properties of the updating process is to look at%\vspace*{-0.8ex}
%\gwc[oder doch als Quotient?]{
\begin{equation}
{\rm PG}:= {\rm MPI}\uz - {\rm MPI}\un \,,%\vspace*{-1.0ex}%
\end{equation}%}
which is called \emph{main parameter precision gain} here.
\end{remark}
Taking the Normal-Normal and the Dirichlet-Multinomial model as concretisations
of Example~\ref{ex:jstp-1}, inference with \ymodel s is now illustrated.

\begin{example}[Normal-Normal Model]
\label{ex:ymodel-nv}
%\noindent\textbf{Example 1a (continued).}
In the Normal-Normal model as presented in Section~\ref{sec:norm-norm},
$\yz$ corresponds to the expected value for $\mu$; the choice of $\YZ$ in application should thus be easy.
To simplify notation, we will assume here and later on that $\sigma^2_0 = 1$.
%Assuming the prior knowledge suggests values for $\mu$ in the range $[3;4]$, we define
%${\cal Y}\uz = [\ul{y}\uz ; \ol{y}\uz] = [3;4]$. For fixing
Let us assume $\YZ = [\yzl ; \yzu] = [3;4]$, and for fixing
$\nz$, suppose further that we are not very certain about
this prior range for $\mu$, but still think it is quite a reasonable
assumption, and so we base it on $5$ pseudo observations by choosing $\nz=5$,
giving a value of the variance for the prior distribution on $\mu$ of $\frac{1}{5}$.
%
%\medskip
%
Updating this prior with the i.i.d.\ sample $\x \in \reals^n$ yields%\vspace*{-0.8ex}%
\begin{align*} %\label{080306-nv_bsp_held}
\ynl &= \frac{\nz\yzl + \sum_{i=1}^n x_i}{\nz + n}, &%\hspace*{-1.5ex}
\ynu &= \frac{\nz\yzu + \sum_{i=1}^n x_i}{\nz + n}, &%\hspace*{-1.5ex}
\nn &= \nz + n\,.%\vspace*{-3.1ex}%
\end{align*}
%
\begin{figure}
%\fbox{%
\caption{Prior (left) and posterior (right) credal sets for a sample
from $\norm(\mu,1)$ drawn as sets of normal cdfs. (Example~\ref{ex:ymodel-nv} in the
situation of no prior data conflict.)} \label{fig:nv-nfest-nopdc}
\includegraphics*[height=\textwidth,angle=-90,bb = 165 60 440 765]{fig/jstp-paper_nv_nfest_01-080331.ps}%
%\includegraphics*[\textwidth]{fig/jstp-paper_nv_nfest_01-080331}%
%}
\end{figure}%
To make this concrete, consider a sample of size $n = 10$
with $\ttau(\x) = \bar{x} = 4$.
Then $\YN = [\frac{55}{15} ; \frac{60}{15}] \approx [3.67 ; 4]$,
${\rm MPI}\un=\frac{1}{3}$ and $\nn = 15$. The posterior credal set
consists therefore of all convex combinations of normal
distributions with means in $[3.67 ; 4]$ and variance
$\frac{1}{15}$.
%
% ------------ verschoben anfang
%
Prior and posterior beliefs can be illustrated by the union of
credal intervals calculated as highest density (HD) intervals%
\footnote{See the concept of highest posterior density (HPD) intervals
as mentioned in Section~\ref{sec:beta-binom}, which used here also to
illustrate the prior state of knowledge.}
%HD intervals\footnote{For a given distribution, a HD interval
%(for \emph{h}ighest \emph{d}ensity) gives a set
%of most plausible values identified as the set %with the shortest range
%of values with highest density
%resulting / aggregating a certain amount of probability weight $\gamma$.}
for all distributions in the corresponding credal set.
As the normal distributions with mean $\yz \in \YZ$ are the extreme points of the prior credal set,
and the normal distributions are stochastically ordered with respect to the mean,
the prior union is the interval from the lowest lower border of HD intervals
(calculated from $\norm(\yzl, \frac{1}{\nz})$) to the highest upper border
(calculated from $\norm(\yzu, \frac{1}{\nz})$).
For a probability weight $\gamma = 0.95$, we get $[2.123;\, 4.877]$.
%
% ------------ verschoben ende
%
The posterior union of HD intervals is
$[3.161;\, 4.506]$ and, covering a much smaller range as a priori, shows
the decreasing of uncertainty obtained by the update step, also
reflected in a main parameter precision gain of ${\rm PG}=\frac{2}{3}$.
This update step is illustrated in Figure~\ref{fig:nv-nfest-nopdc},
where the prior and posterior credal set are displayed by the normal
cumulative distribution functions, the black lines indicating the
functions defined by the vertices of $\YZ$ and $\YN$, respectively. %\sidenote{statt: that constitute their vertices.}
The observation $\ttau(\x)$ is marked by the point of the triangle
in both graphs, and the prior and posterior union of HD intervals are marked
by a thick line in the graph for the prior and posterior set, respectively.
\end{example}

\begin{example}[Dirichlet-Multinomial Model]
\label{ex:ymodel-idm}
An \ymodel\ based on the Dirichlet-Multinomial Model as discussed in Section~\ref{sec:diri-multi}
is, for $\YZ = \Y$, equivalent to the imprecise Dirichlet model
(IDM, see Section~\ref{sec:idm-and-near-ignorance}), and was considered in Section~\ref{sec:fixedlearningparameter}
for the common-cause failure application.

In the usual applications of the IDM, the aim is to start with prior
ignorance; this is modelled by choosing $\YZ$ as the unit
simplex. For $\nz$ values of 1 or 2 are suggested. Here,
we must rely on the interpretation of $\nz$ as prior strength, as
there is no interpretation in terms of other
parameters as in Example~1a.
%
%\medskip
%
Considering prior knowledge for a three-category multinomial model
suggesting that extreme values for $\theta_1$ and $\theta_2$ are
implausible, one could choose
$\YZ = \{ \yz_1 \in [0.2;0.8] \times \yz_2 \in [0.2;0.8] \times \yz_3 \in [0;0.6]\}$,
where the upper bound for $\yz_3$ is a result of the unit
simplex constraint $\sum_{j=1}^k \yz_j = 1$.
In addition, we choose again $\nz=5$ as in Example~\ref{ex:ymodel-nv}.
%
%\medskip
%
Considering a sample of size $5$, where $3$ observations are of
category 1, and $2$ of category 2, we get $\nn = 10$ and the ranges
$\yn_1 \in [0.4 ; 0.7]$,
$\yn_2 \in [0.3 ; 0.6]$, and
$\yn_3 \in [0   ; 0.3]$
for the posterior class probabilities.
%For instance, the posterior
%probability for observing category two or three in the next draw
%is, by applying conjugacy, between $0.3$ and $0.6$.
In analogy to Example~\ref{ex:ymodel-nv}, this update step is illustrated with the
left and center graph of Figure~\ref{fig:idm-nfest-nopdc}, where
prior and posterior credal sets are represented by cutouts from a
plane in the three-dimensional parameter space. Each point in the
plane cutout for the prior set on the left graph represents a
certain combination of $\yz_1$, $\yz_2$, and $\yz_3$ by the
magnitude of coordinates. The same applies for the posterior set
depicted in the center graph. Some additional lines were drawn to
make locating the cutouts in space more easy.
%
\begin{figure}%\hspace*{3.5ex}%
%\begin{center}%
%\fbox{%
\caption{Prior (left) and posterior (center, right) credal sets for samples from
$\mult(\btheta)$ in accordance with (center) and, as studied in
Section~\ref{section:fixednschlecht}, contrary to (right) prior
beliefs. Note that both posterior sets have the same shape and size and differ
only in location, in contrast to the ones depicted in Figure~\ref{fig:idm-nvar-nopdc}.}
\begin{tabular}{ccc}%
\hspace*{-1.2ex}%\fbox{%
\includegraphics*[height=0.33\textwidth,angle=-90,bb = 100 65 520 385]{fig/jstp-paper_idm_nfest_01-080331.ps}%
%}
\hspace*{-1.2ex}%
&%
\hspace*{-1.2ex}%\fbox{%
\includegraphics*[height=0.33\textwidth,angle=-90,bb = 100 470 520 790]{fig/jstp-paper_idm_nfest_01-080331.ps}%
%}
\hspace*{-1.2ex}%
&%
\hspace*{-1.2ex}%\fbox{%
\includegraphics*[height=0.33\textwidth,angle=-90,bb = 100 470 520 790]{fig/jstp-paper_idm_nfest_02-080331.ps}%
%}
\hspace*{-1.2ex}%
\end{tabular}%
%}%
%Illustration of an update step for Example~1b (IDM / \ymodel).}
\label{fig:idm-nfest-nopdc}
%\end{center}%
%\hfill
\end{figure}%
\end{example}



\section{***Software***}

seperate section, or integrate into jstp?

\section{***Isipta'07 paper***}

seperate section, or integrate into jstp?

