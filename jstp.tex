\section{***Generalised iLUCK-models (JSTP paper)***}
\label{sec:jstp}

****short intro from intro and abstract, examples already in Chapter~\ref{cha:intro}


\medskip

***The section is structured as follows:
In Section~\ref{070517-sec2-1} we provide the formal background
by distinguishing a wide class of classical, precise probability models
where Bayesian inference has a particular form, being directly
suitable to a generalization to imprecise probabilities by varying
the linearly updated main parameter (Section~\ref{sec:3}).
We illustrate these models and then show in Section~\ref{sec:4-gw-071216}
how these models can be extended to deal with prior-data conflict in a more sensible way
by additionally varying another parameter.
Section~\ref{sec:illu} illustrates our procedure in two important special cases:
inferences on the mean of a normal distribution and in the IDM.***


\subsection{Traditional Bayesian inference and LUCK-models}\label{070517-sec2-1}

%\section{Bayesian Inference and \textsc{luck}-models}\label{070516-sec2}
%\subsection{Classical Bayesian Inference and \textsc{luck}-models}\label{070517-sec2-1}
In most cases, traditional Bayesian inference is based on so-called
conjugate priors (related to a specific likelihood). These
distributions have the convenient property that the posterior
resulting from~\eqref{eq:bayesrule} belongs to the same class of parametric
distributions as the prior. The posterior thus remains easily
tractable, and updating can be described in terms of parameters
only. For describing the imprecise probability model used later on more easily, we want to
%For the intended application presented later on, it is quite convenient to
distinguish certain standard situations (called
\emph{models with `Linearly Updated Conjugate prior Knowledge'
(LUCK)} here) of Bayesian updating with classical (traditional, precise)
probabilities, where prior and posterior fit nicely together in the sense that
(i) they belong to the same class of parametric distributions, and, in addition,
(ii) the updating of one parameter ($\yz$ below) of the prior is
linear given a second parameter ($\nz$).
%\begin{enumerate}%
%\vspace*{-1.2ex}%
%\item[i)] they belong to the same class of parametric distributions,
%and, in addition,\vspace*{-1.2ex}
%\item[ii)] the updating of one parameter ($y^{(0)}$ below) of the prior is
%linear given a second parameter ($n^{(0)}$).
%\end{enumerate}
%\vspace*{-1.2ex}%
More precisely, we return to the following definition,
originally introduced in \textcite{Walter2007a}:%\vspace*{-0.5ex}
\footnote{Compare to the model presented in Section~\ref{sec:regularconjugates}.
Here, the likelihood $f(\x\mid\vartheta)$ may have any form,
given that the update step from prior to posterior distribution
adheres to Equations~\eqref{070305-2} -- \eqref{070305-4}.
The model discussed in Section~\ref{sec:regularconjugates} is a special case,
as Equations~\eqref{070305-2} and \eqref{070305-4} are the same as
Equations~\eqref{eq:canonicalprior} and \eqref{eq:canonicalupdate}, respectively.
See Example~\ref{ex:jstp-1} below.}

\begin{definition}[LUCK-models]\label{070503-defin1}
Consider traditional Bayesian inference on a parameter $\vartheta$
based on a sample $\x$ as described in Section~\ref{sec:bayes-inference} (Equation~\eqref{eq:bayesrule}),
and let the prior $p(\vartheta)$ be characterized by the (vectorial) parameter
$\vartheta\uz$. The pair $\big(p(\vartheta),
p(\vartheta\mid\x)\big)$ is said to constitute a
\emph{LUCK-model of size $n\in \naturals$ with respect to the
likelihood $f(\x\mid\vartheta)$ in the natural parameter $\psi$ with
prior parameters $\nz \in \posreals$ and $\yz$ and sample
statistic $\tau(\x)$} iff $p(\vartheta)$ and $p(\vartheta\mid\x)$
can be rewritten in the following way:%
%\footnote{$\langle a,b \rangle$ denotes the scalar product of $a$ and $b$.}%\vspace*{-1.2ex}
\begin{equation}\label{070305-2}
p(\vartheta) \propto \exp\big\{ \nz \big[\langle \psi, \yz \rangle - \mbf{b}(\psi)\big]\big\} %\vspace*{-1.5ex} %\label{equ:QdCpriori}
\end{equation}%
and %\vspace*{-1.0ex}
\begin{equation}\label{070305-3}
p(\vartheta\mid\x) \propto \exp\big\{ \nn \big[\langle \psi, \yn \rangle - \mbf{b}(\psi)\big]\big\}\,,%\vspace*{-1.5ex} %\label{equ:QdCpriori}
\end{equation}
with
\begin{align}\label{070305-4}
%n^{(1)} &= n^{(0)}+n & &\mbox{and}  & y^{(1)} &= \frac{n^{(0)}y^{(0)}+\tau(x)}{n^{(0)}+n}\,,\vspace*{-3.5ex}  %\label{equ:QdCupdate}
\yn &= \frac{\nz}{\nz + n} \cdot \yz + \frac{n}{\nz + n} \cdot \frac{\tau(\x)}{n}\,, &
\nn &= \nz + n\,,
\end{align}%\vspace*{-1.0ex}%
where $\vartheta$ is transformed to $\psi$ and $\mbf{b}(\psi)$,
and $\vartheta\uz$ to $\nz$ and $\yz$.
\end{definition}%\vspace*{-0.5ex}%\vspace*{-1.9ex}%
%
$\yz$ and $\yn$ can be seen as the parameter describing
the main characteristics of the prior and the posterior,
respectively, and so later on, $\yz$ and $\yn$ will be
called \emph{main prior} and \emph{main posterior parameter}.
In the models considered here, $\yz$ can also be understood as a prior
guess for the random quantity $\ttau(\x) := \tau(\x)/n$
summarizing the sample. According to the left
part of \eqref{070305-4} these two different sources of information
are linearly combined to obtain the main posterior parameter:
%\vspace*{-1.3ex}
\begin{equation}\label{071214-2}
\yn = \frac{\nz}{\nz + n} \cdot \yz + \frac{n}{\nz + n} \cdot \ttau(\x)\,. %\vspace*{-1.0ex}%\frac{1}{q}\tau(w).
\end{equation}
This relation also equips $\nz$ with a vivid interpretation as ``prior strength''
or as ``pseudocounts'',
%which will become clearer in Section~\ref{070514-secImpPriorLUCK-QdC}.
reflecting the weight one gives to the prior with respect to the sample.
So, $\nz$ can be
interpreted as the size of an imaginary sample that corresponds to
the trust on the prior information in the same way as the sample
size of a real sample corresponds to the trust in conclusions based
on such a real sample. %
%
%\altrev{This interpretation is supported by the left part of
%(\ref{070305-4}): If in consecutive Bayes learning the posterior is
%used as the new prior, then the new main prior parameter $y\uo$ gets
%prior strength $n\uo = n\uz + n$.}%\gwc[mehr zu pseudocounts in BernardoSmith?]{}
%\gwc[hier schon was zu vektoriellen Größen?]{}
%\sidenote{Absatz zu $\vartheta\uz$ vectorial auskommentiert}
%where consistently throughout the paper vectors are evaluated component
%by component, here possibly leading to a vector-valued quantity.
%
%If $\vartheta\uz$ is vectorial, it is possible that $y\uz$ and
%$\tilde{\tau}(w)$ are vectors as well. In this case all functionals
%of vectors in this paper \thccc{are} evaluated component by
%component, leading to a vector-valued quantity.
%
As a preparation for the generalizations considered later, let us
turn to some characteristic examples, also illustrating the
interpretations of $\yz$ and $\nz$.

\begin{example}[Bayesian Inference in Exponential Families]
\label{ex:jstp-1}
%\noident\textbf{Example 1: Bayesian Inference in Exponential Families.}
In the case of independently and identically
distributed (i.i.d.) observations $\x = (x_1,\ldots, x_n)$ from
\emph{regular canonical exponential families} \parencite[p.~202 and p.~272f]{2000:bernardosmith},
a general result (see, e.g., ibid., Proposition~5.4) is available on how
to construct conjugate priors. A prior obtained by this method then
constitutes a LUCK-model of size $n$ (the sample size)
with the sample statistic of the whole sample being the sum of
statistics for each sample element (which can be concluded from the
canonical form of the likelihood), so $\tau(\x) = \sum_{i=1}^n \tau^*(x_i)$,
and $\ttau(\x) = \frac{1}{n}\sum_{i=1}^n \tau^*(x_i)$.%
\footnote{\textcite{2005:quaeghebeurcooman} consider this special case of LUCK-models in
their seminal work motivating the generalisations presented here.}
This is effectively the framework of Bayesian inference with regular conjugate priors
as presented in Section~\ref{sec:regularconjugates}.
To perceive the generality of this result, recall that many of the
sample models most often used in practice form an exponential family,
as shown earlier in this thesis
for the Binomial distribution (Section~\ref{sec:beta-binom}),
the Normal or Gaussian distribution (Section~\ref{sec:norm-norm}),
and the Multinomial distribution (Section~\ref{sec:diri-multi}).
%For samples from a scaled normal and from a
%multinomial distribution, the derived priors are presented explicitly here:
\end{example}

\iffalse
\noindent\textbf{Example 1a: Samples from a scaled Normal $\mbox{N}(\mu,1)$.}
The conjugate distribution to an i.i.d.-sample of size $n$ from a
scaled normal distribution with mean $\mu$, denoted by $\mbox{N}(\mu,1)$,
as constructed by this method is a normal distribution with
mean $y\uz$ and variance $\frac{1}{n\uz}$. Here, we have $\psi = \mu$,
$\mbf{b}(\psi) = \frac{\mu^2}{2}$, and $\tau(x_i) = x_i$,
thus $\tilde{\tau}(x) = \bar{x}$, and\vspace*{-1.5ex}
\begin{eqnarray*}
p(\mu) &\propto& \exp \left\{ -\frac{n\uz}{2} \left(\mu - y\uz\right)^2 \right\}
        \propto  \exp \left\{ n\uz \left[ \mu \cdot y\uz - \frac{\mu^2}{2} \right] \right\}\,.
%      \,\hat{=}\, \exp \big\{ n\uz \big[ \langle \psi, y\uz \rangle - \mbf{b}(\psi) \big] \big\}\,.
\end{eqnarray*}
%\begin{eqnarray*}
%\prod_{i=1}^n p(w_i\,|\,\mu) &=& \prod_{i=1}^n \frac{1}{\sqrt{2 \pi }} \exp \{ -\frac{1}{2}(w_i - \mu)^2 \} \\
%                             &=& \frac{1}{(2 \pi)^\frac{n}{2}} \exp \{ -\frac{1}{2} \sum_{i=1}^n (w_i - \mu)^2 \}
%\end{eqnarray*}
%\medskip
\textbf{Example 1b: Samples from a Multinomial $\mbox{M}(\btheta)$.}
Given a sample of size $n$ from a multinomial distribution with
probabilities $\theta_j$ for categories $j= 1,\ldots,k$, subsumed in
the vectorial parameter $\btheta$ (with $\sum_{j=1}^k \theta_j =1$),
the conjugate prior on $\btheta$ constructed along the general
method \thcc{can be shown to be} the Dirichlet distribution
$\mbox{Dir}(\bs{\alpha})$. Written in terms of
Walley's \citeyearpar{Walley-IDM} reparameterization
$\alpha_j = s\cdot t_j$ as used later on, it holds that $n\uz = s$
and $y\uz = \mbf{t}=(t_1,\ldots,t_k)^\tra$. Note in
particular that here the components of $y^{(0)}$ have a direct
interpretation as class probabilities. This so-called
Dirichlet-Multinomial model provides the basis of the Imprecise
Dirichlet Model (IDM), introduced by \cite{Walley-IDM}, that is
considered in the continuation of this example in Section~\ref{sec:3}.%
\fi

\begin{example}[Bayesian Inference in Linear Regression]
%\noindent\textbf{Example 2: Bayesian Inference in Linear Regression.}
As shown by \textcite{Walter2006a,Walter2007a,Walter2007b},
the importance of LUCK-models is not limited to the i.i.d.\ case
but also provides a formal superstructure containing in particular
the practically important case of linear regression
models, modelling the (linear) influence of certain variables (called covariates,
confounders, regressors, stimulus or independent variables) on a
certain outcome (also called response or dependent variable).%
\footnote{See Chapter~\ref{cha:festschrift} (Appendix~1) for alternative models
that adhere to the regular conjugate framework of Section~\ref{sec:regularconjugates}.}
\end{example}


\subsection{Imprecise priors for inference in LUCK-models}\label{sec:3}%\label{070514-secImpPriorLUCK-QdC}

Our definition of LUCK-models was inspired by the work of
\textcite{2005:quaeghebeurcooman}, who develop a general approach for inference with
imprecise priors, which was proven in \textcite{Walter2007a} to be
generalisable to arbitrary LUCK-models. Quaeghebeur
and de Cooman's seminal idea was that the seemingly strange
parameterization in terms of $\yz$ and $\nz$ in
\eqref{070305-2} and \eqref{070305-3} is perfectly suitable to be
generalized to credal sets of priors. The crucial point is that
$\yz$, the main prior parameter%
%\footnote{Example 1a: the prior mean for $\mu$;
%Example 1b: the vector of prior expected values for probabilities $\btheta$}
\footnote{In the Normal-Normal model: the prior mean for $\mu$;
in the Dirichlet-Multinomial model: the vector of prior expected values
for the category probabilities $\btheta$.}
is updated \emph{linearly}, when
the prior strength $\nz$ is taken as fixed. This makes an
easily tractable imprecise calculus possible: When sets of priors
are defined via sets of main parameters $\yz$, and these sets of
$\yz$ are defined by lower and upper bounds, the lower and upper
bounds of the sets of posterior main parameters $\yn$ can be
obtained directly from \eqref{070305-4}.





\section{***Software***}

seperate section, or integrate into jstp?

\section{***Isipta'07 paper***}

seperate section, or integrate into jstp?

