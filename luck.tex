\section{The \texttt{luck} Package}
\label{sec:luck}

%seperate section, take short description from abstract for ISIPTA'09 poster?

%Section~\ref{sec:luck} gives a short overview on a software implementation of \nymodel s,
%the add-on package \texttt{luck} (citation) for the statistical programming environment \textbf{R} \parencite{2013:r}.

This secion gives a short overview on a software implementation of the models discussed in Section~\ref{sec:jstp},
the add-on package \texttt{luck} \parencite{luck-package} for the statistical programming environment \textbf{R}.
Author and maintainer of this package is Gero Walter,
with Norbert Krautenbacher as a second author, who contributed code for an application
to exponentially distributed data.

\paragraph{Introduction.}
In recent years, the statistical programming environment %software
\textbf{R} \parencite{2013:r} is used more and more frequently for every-day statistical analyses
in academia and inside corporations. %(Wikipedia)
It has also become a de facto standard among
statisticians for software implementations of novel methods,
which can be easily distributed by means of so-called \emph{packages}
%, additional functionality can be provided to \texttt{R} users
via the online repository `cran' (Comprehensive \textbf{R} Archive Network, \url{http://www.cran.R-project.org}).
%
%implements the general model structure from Section~\ref{sec:jstp}.
The present implementation is programmed in a class system of \textbf{R}
that reflects %mirrors
the hierarchy between the unified description in terms of \model s%
\footnote{As described in footnote~\ref{foot:luckmodels}, page~\pageref{foot:luckmodels},
\model s generalise the framework
of canoncial conjugate priors from Section~\ref{sec:regularconjugates},
by requiring only the form of the update step \eqref{eq:canonicalupdate},
and not the specific functional form of \eqref{eq:expofam-sampledens} for $f(\x \mid \vartheta)$
(see Definition~\ref{070503-defin1}).}
and the concrete application of this framework by, e.g.,
considering inference on the mean of scaled normal distribution using sets of conjugate priors
(as described in Example~\ref{ex:ymodel-nv}).
%making extensions for arbitrary prior classes very easy.
This hierarchical structure makes extensions of the package in order to
enable inference in a wide class of sample distributions very easy.

\paragraph{Object-oriented Programming.}

Such hierarchical structures 
can be implemented using the object-oriented paradigm for programming.
The central tools in this paradigm are \emph{classes} and \emph{methods}.
\emph{Classes} are used to represent general concepts that should be modelled in software.
Such a concept is structured through a class by defining the traits that
examples for the general concept should have. 
Classes thus provide the blueprint for \emph{objects},
by defining a number of \emph{attributes} or \emph{slots} these objects have.
Given concrete values for the slots of the class,
a concrete object can be created according to the blueprint provided by the class definition.
%which concrete objects can be created.
These concrete realisations of a class are called \emph{instances}.
\emph{Methods} then provide functions to manipulate such instances,
the class description guaranteeing a standardised input for the functions.
However, the most prominent feature of the object-oriented paradigm is
that hierarchies between concepts can be modelled by \emph{inheritance}.
More specific concepts can be modelled as special cases of a general concept, % behind more general classes,
and the blueprint for such specialised objects are called \emph{subclasses},
which \emph{inherit} the traits of the more general class.
Moreover, also methods can be specialised to reflect the class hierarchies,
by giving more specific output for subclasses.
The relations between classes are usually depicted in so-called \emph{UML graphs}.
Figure~\ref{fig:uml-general} gives an example for such a graph.

\begin{figure}
\centering
\begin{tikzpicture}[class/.style={draw, rectangle split, rectangle split parts=3, %
                                  every text node part/.style={text centered}, %
                                  fill=gray!30, text width=53mm, thick}, %
                    point/.style={coordinate}]%every node/.style=draw]
%\draw[help lines] (0,0) grid (2,2);
\node[class] (superclass) at (1,1) {Some Class \nodepart{second} `slots' (attributes) \nodepart{third} `methods' (functions)};
\node[class] (subclass) [below=0.75cm of superclass] {Subclass \nodepart{second} \small additional/specialised slots
                                                               \nodepart{third}  \small additional/specialised methods};
\node [right=0.5cm of subclass, draw=none] {\parbox{25ex}{`Subclass' extends\\ `Some Class', inheriting\\ slots and methods}};
\draw [stealth'-, thick] (superclass.south) -- (subclass.north); %node [midway, right, draw=none] {extends Some Class; inherits slots \& methods};
\node[class, text width=45mm] (subclass2) [below=0.8cm of subclass] %
{\small Subsubclass 2 \nodepart{second} \scriptsize additional/specialised slots
                      \nodepart{third}  \scriptsize additional/specialised methods};
\draw [stealth'-, thick] (subclass.south) -- (subclass2.north) node (p0) [coordinate, midway] {}; %node [midway,right,draw=none] {inherits};
\node[class, text width=45mm] (subclass1) [left=0.5cm of subclass2] %
{\small Subsubclass 1 \nodepart{second} \scriptsize additional/specialised slots
                      \nodepart{third}  \scriptsize additional/specialised methods};
\node[class, text width=45mm] (subclass3) [right=0.5cm of subclass2] %
{\small Subsubclass 3 \nodepart{second} \scriptsize additional/specialised slots
                      \nodepart{third}  \scriptsize additional/specialised methods};
\node [coordinate, above=0.4cm of subclass1] (p1) {};
\draw [thick] (p0) --  (p1) -- (subclass1.north);
\node [coordinate, above=0.4cm of subclass3] (p3) {};
\draw [thick] (p0) --  (p3) -- (subclass3.north);
\end{tikzpicture}
\caption[Illustration of class hierarchies in object-oriented software (UML diagram).]%
{Illustration of class hierarchies in object-oriented software.
Each class is drawn as a rectangle with three parts,
the top part giving the name of the class.
The two lower parts name the slots and the methods for the class, respectively.
A class that inherts from another class is linked to this other class with an arrow.
Such graphs are called \emph{UML diagrams}.}
\label{fig:uml-general}
\end{figure}

\paragraph{The Basic Classes.} %{Basic Structure of the Package.}

The package \texttt{luck} uses the S4 class system of
%To recap, the present implementation is programmed in the S4 class system of
\textbf{R}, providing the basic framework for the definition, display and updating of
sets of priors based on \model s by defining the central class \texttt{LuckModel}.
Inferences in a concrete sample distribution are then carried out
using lean subclasses that make the `translation'
between the (abstract) \model\ framework and a concrete sample distribution.

For defining prior parameter sets $\PZ$,
the class \texttt{LuckModel} provides the slots \texttt{n0} for $\nz$, and \texttt{y0} for $\yz$, respectively,
the contents of which are defined internally as intervals,
but where the lower and upper bound may coincide,
such that both $\NZ$ and $\YZ$ can be either an interval or a singleton.%
\footnote{In case of $\byz \in \reals^k$, $\YZ$ may be a multidimensional interval,
i.e., a cartesian product of intervals $[\yzl_j,\yzu_j]$, $j=1,\ldots,k$,
(see Example~\ref{ex:ymodel-idm}),
with $\yzl$ and $\yzu$ denoting the vectors of these element-wise lower and upper bounds, respectively.} 
From the model types discussed in Section~\ref{sec:basicsetting}, type
(\ref{enum:modeltypes-a}), (\ref{enum:varyn}), and (\ref{enum:rectangular})
can thus be implemented by choosing \texttt{n0} and \texttt{y0} accordingly
as singletons or intervals.
%
To calculate the set of posterior distributions,
the class \texttt{LuckModel} also provides a \texttt{data} slot.
This must contain an object of class \texttt{LuckModelData},
providing the data in the needed form ($\tau(\x)$ and $n$).

\paragraph{Posterior Parameter Sets.}

As the posterior parameter sets $\PN$ resulting from updating all the priors in a
prior parameter set $\PZ$ are, in the most general case of $\PZ = [\nzl,\nzu] \times [\yzl, \yzu]$,
%\footnote{In the case of multidimensional $\byz$,
%$\ynl$ and $\ynu$ are understood as element-wise lower and upper bounds of $\YN$, respectively.}
not cartesian products of $[\nnl, \nnu]$ and $[\ynl, \ynu]$ anymore, posterior sets are not
explicitely represented as \texttt{LuckModel} objects. Whenever posterior quantities
are of interest (specified in functions or methods for \texttt{LuckModel} objects by the option \texttt{"posterior = TRUE")},
the range of these quantities are calculated by minimising and maximising over $\PN$, %the updated parameters,
which in turn can be done by a box-constrained optimisation over $\PZ$ %the set of prior parameters, as $\PZ$
that is a cartesian product.
This is why the data object (\texttt{LuckModelData}) is directly included in the \texttt{LuckModel} object.
For the box-constrained optimisation, a helper function called \texttt{wrapOptim} is used,
such that all cases (none, one or both of \texttt{n0} and \texttt{y0} %canonical parameters
are interval-valued) can be treated in the same way.%
\footnote{The function \texttt{optim} provided by \textbf{R} for general-purpose
multivariate optimisation is not recommended for univariate optimisation.}

\paragraph{Subclasses for Concrete Distributions.}

As mentioned above, the class \texttt{LuckModel} in fact only implements the general superstructure as given in the
(abstract) definition of \model s (see Definition~\ref{070503-defin1}).
For inference in a concrete distribution family like the Normal distribution,
this general framework must be concretised %. This can be done
by defining a subclass, inheriting from \texttt{LuckModel},
for this specific distribution family,
along with a subclass of \texttt{LuckModelData} to specify how $\tau(\x)$
is calculated for this distribution family.
Currently, this has been done for data from a scaled normal distribution,
i.e., $x_i \sim \norm(\mu, 1)$,%
\footnote{See Example~\ref{ex:ymodel-nv}, p.~\pageref{ex:ymodel-nv}f.}
with the classes \texttt{ScaledNormalLuckModel} and \texttt{ScaledNormalData},
and for data from an exponential distribution, i.e., $x_i \sim \text{Exp}(\lambda)$,%
\footnote{See the study by \textcite{2011:krautenbacher},
who contributed the code for this distribution family.
The results of this study are briefly discussed in Section~\ref{sec:alternatives:whitcomb}.}
with the classes \texttt{ExponentialLuckModel} and \texttt{ExponentialData}.
Figure~\ref{fig:uml-luck} shows the UML diagram,
illustrating the hierarchical structure of the package. 

\begin{figure}
\centering
\begin{tikzpicture}[class/.style={draw, rectangle split, rectangle split parts=3, %
                                  every text node part/.style={text centered}, %
                                  font=\ttfamily, %
                                  fill=gray!30, text width=45mm, thick}, %
                    point/.style={coordinate}]%every node/.style=draw]
\node [class] (luck) at (0,0) {LuckModel \nodepart{second} n0: matrix \\
                                         y0: matrix \\
                                         data: LuckModelData
                                         \nodepart{third}
                                         show() \\
                                         plot() \\
                                         unionHdi() \\
                                         \quad \vdots};
\node [coordinate, below=0.4cm of luck] (p0) {};
\node [coordinate, left=0.4cm of p0] (p1) {};
\node [class, text width = 47mm] (scalednormal) [below=0.4cm of p1] {ScaledNormalLuckModel\phantom{p} %
                                                                        \nodepart{second} \phantom{pS}
                                                                        \nodepart{third} singleHdi()\phantom{p}\\};
\draw [stealth'-, thick] (luck.south) -- (p0) -- (p1) -- (scalednormal.north); %node [midway,right,draw=none] {inherits};
\node[class, text width=45mm] (poisson) [right=0.5cm of scalednormal] {ExponentialLuckModel %
                                                                        \nodepart{second} \phantom{pS}
                                                                        \nodepart{third} singleHdi()\phantom{p}\\};
\node[class, text width=35mm, double copy shadow={shadow xshift=1mm,shadow yshift=-1mm}] %
 (more) [right=0.5cm of poisson] {\phantom{pS}\ldots\phantom{pS} \nodepart{second} \phantom{pS}
                                                                 \nodepart{third} singleHdi()\phantom{pS}\\};
\node [coordinate, above=0.4cm of poisson] (p2) {};
\draw [thick] (p0) -- (p2) -- (poisson.north);
\node [coordinate, above=0.4cm of more] (p3) {};
\draw [thick] (p2) -- (p3) -- (more.north);
%\draw[->] (0,0) -- (1,0);
\begin{scope}[transform canvas={scale=0.75}]
%\draw[->] (0,0.5) -- (1,0.5);
%\node [coordinate] (lmd) [right=0.5cm of luck] {}; 
\node [class] (data) at (6.8,1.5) {LuckModelData \nodepart{second} tauN: matrix\\ rawData: matrix
                                                             \nodepart{third} show()};
\node [coordinate, below=0.4cm of data] (dp0) {};
\node [coordinate, left=0.4cm of dp0] (dp1) {};
\node[class, text width=37mm] (scalednormaldata) [below=0.4cm of dp1] {ScaledNormalData\phantom{p} \nodepart{second} \phantom{h}
                                                                                                   \nodepart{third} show()};
\node[class, text width=37mm] (poissondata) [right=0.5cm of scalednormaldata] {ExponentialData \nodepart{second} \phantom{h}
                                                                                               \nodepart{third} show()};
\node[class, text width=30mm, double copy shadow={shadow xshift=1mm,shadow yshift=-1mm}] %
 (moredata) [right=0.5cm of poissondata] {\phantom{pD}\ldots\phantom{pD} \nodepart{second} \phantom{h}
                                                                         \nodepart{third}  show()};
\draw [stealth'-, thick] (data.south) -- (dp0) -- (dp1) -- (scalednormaldata.north); %node [midway,right,draw=none] {inherits};
\node [coordinate, above=0.4cm of poissondata] (dp2) {};
\draw [thick] (dp0) -- (dp2) -- (poissondata.north);
\node [coordinate, above=0.4cm of moredata] (dp3) {};
\draw [thick] (dp2) -- (dp3) -- (moredata.north);
\end{scope}
\node [coordinate] (snake) at (2.75,1.025) {};
\draw [dashed, very thick, -stealth'] (2.5,0.35) .. controls +(right:4mm) and +(up:4mm)
                                      .. (snake) .. controls +(up:4mm) and +(left:8mm) .. (3.2,1.7);
\end{tikzpicture}
\caption[UML diagram for the \texttt{luck} package, illustrating the class hierarchy.]%
{UML diagram for the \texttt{luck} package, illustrating the class hierarchy.
In UML diagrams, the slots and methods that
subclasses inherit are not specify explicitely;
only new slots and methods are displayed.
The class of each slot is given next to its name.
The \texttt{data} slot in \texttt{LuckModel} is of class \texttt{LuckModelData},
indicated by the dashed arrow.}
\label{fig:uml-luck}
\end{figure}

\paragraph{Implemented Methods.}

For illustrations of \texttt{LuckModel} objects and inferences based on them, some methods have
been written. First, there are methods to display and print plain \texttt{LuckModel}
objects (existing only on the superstructure level):
\begin{itemize}
\item So-called \emph{constructor functions} for the \texttt{LuckModel} and \texttt{LuckModelData} class
   make the creation of \model s more easy. They can be supplied with a 
   number of different arguments, which are checked on consistency upon
   creation of the object.
\item Due to the S4 implementation, so-called \emph{accessor} and \emph{replacement functions}
   are defined, regulating the access and the replacement of object slots.
\item The \texttt{show} method prints the contents of a \texttt{LuckModel} object in more
   readable form. If the object contains a \texttt{LuckModelData object}, this is
   printed along as well, resorting on a \texttt{show} method for \texttt{LuckModelData}
   objects.\footnote{\texttt{show} methods are the S4 equivalent to \texttt{print} methods for S3 objects.}
\item The \texttt{plot} method for \texttt{LuckModel} objects represents the parameter sets graphically.
\end{itemize}
Secondly, there are methods for working with
and displaying the resulting sets of prior or posterior distributions
for concrete data distributions, along with two exemplary inference methods.
\begin{itemize}
\item The constructor functions are modified, for example to check if the input
   fits to the data or prior distribution, or to allow the simulation of data
   according to the data distribution when creating the \texttt{LuckModelData} object.
\item The accessor and replacement functions need not be specified again for the
   specialized classes, as those can be taken from the general classes, i.e.,
   these functions are 'inherited' from the respective \texttt{LuckModel} or \texttt{LuckModelData}
   class. An exception is the function for replacing the raw data in the
   \texttt{LuckModelData object}, as there, the input must be checked to fit the sample
   space domain. As an example, in case of the \texttt{ExponentialData} class, data must be
   strictly positive.
\item The \texttt{show} methods for \texttt{LuckModel} and \texttt{LuckModelData} are specialised in order
   to explain to the user the meaning of the canonical parameters $\nz$ and $\yz$.
\item \texttt{unionHdi} calculates the union of highest density intervals for a
   specialised \texttt{LuckModel} object.%
   \footnote{Highest density (HD) intervals were discussed in Section~\ref{sec:beta-binom},
   while unions of highest density intervals were considered, e.g.,
   in Example~\ref{ex:ymodel-nv}, p.~\pageref{ex:ymodel-nv}f.}
   This method relies on a function \texttt{singleHdi},
   that gives a highest density interval for a single parameter combination
   $(n, y)$ for the respective conjugate prior or posterior distribution.
   Therefore, for each specialised LuckModel class, \texttt{singleHdi} must be
   provided.
\item \texttt{cdfplot} plots the set of cumulative density functions for specialised
   \texttt{LuckModel} objects. Again, this method relies on a function \texttt{singleCdf},
   returning values of the cumulative density function for a single parameter combination $(n, y)$
   for the respective conjugate prior distribution.
\end{itemize}










