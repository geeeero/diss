\documentclass{beamer}

\usetheme{LMU}

\usepackage[utf8]{inputenc}
\usepackage[T1]{fontenc}

\usepackage[english]{babel}
\usepackage{url}
%\usepackage{cite}
%\usepackage{natbib}
\usepackage[backend=bibtex,style=authoryear,dashed=false]{biblatex}
\addbibresource{../bib/eigene.bib}
\addbibresource{../bib/itip-refs.bib}
\addbibresource{../bib/other-refs.bib}
\renewcommand{\bibfont}{\normalfont\scriptsize}
\setlength{\bibhang}{3ex}

\usepackage{amssymb, amsmath, amsfonts, enumerate}
%\usepackage{bbold}
\newcommand\hmmax{0}
\usepackage{bm}
%\usepackage{dsfont}
\usepackage{pxfonts}
\usepackage{xcolor}

\usepackage{tikz}
\usetikzlibrary{%
   arrows,%
   calc,%
   fit,%
   patterns,%
   plotmarks,%
   shapes.geometric,%
   shapes.misc,%
   shapes.symbols,%
   shapes.arrows,%
   shapes.callouts,%
   shapes.multipart,%
   shapes.gates.logic.US,%
   shapes.gates.logic.IEC,%
   er,%
   automata,%
   backgrounds,%
   chains,%
   topaths,%
   trees,%
   petri,%
   mindmap,%
   matrix,%
   calendar,%
   folding,%
   fadings,%
   through,%
   patterns,%
   positioning,%
   scopes,%
   decorations.fractals,%
   decorations.shapes,%
   decorations.text,%
   decorations.pathmorphing,%
   decorations.pathreplacing,%
   decorations.footprints,%
   decorations.markings,%
   shadows}
%\usepackage{bbold}
\usepackage{hyperref}

\setbeamertemplate{blocks}[rounded][shadow=true]
\definecolor{lmugreen2}{RGB}{0,120,94}
%\definecolor{lmugreen2}{RGB}{0,148,64}
\definecolor{lmugreen}{RGB}{0,140,84}
\definecolor{unidurham}{RGB}{126,49,123}

\parindent0pt
\setlength{\unitlength}{1ex}
\setlength{\fboxsep}{0ex}

\def\then{{\color{lmugreen}$\rule[0.35ex]{2ex}{0.5ex}\!\!\!\blacktriangleright$}}
%\def\then{{\color{lmugreen}$\blacktriangleright\!\blacktriangleright$}}
%\def\then{{\color{lmugreen}$\blacktriangleright$}}
%\def\then{{\color{lmugreen}$\Rrightarrow$}}
%\def\then{{\color{lmugreen}$\rhd$}}
%\def\then{{\color{lmugreen}$\gg\!\!\!\!\!\gg$}}
%\def\then{{\color{lmugreen}${\mathbf{\gg}}$}}
\def\play{{\color{lmugreen}$\blacktriangleright$}}

\def\rthen{{\color{lmugreen}$\rule[0.35ex]{0.5ex}{0.95ex}\rule[0.35ex]{1.3ex}{0.5ex}\!\!\!\blacktriangleright$}}

\def\thenthen{{\color{lmugreen}$\blacktriangleleft\!\!\!\rule[0.35ex]{2ex}{0.5ex}\!\!\!\blacktriangleright$}}

\def\gplus{{\color{lmugreen}\rule[0.45ex]{1.4ex}{0.4ex}\hspace{-0.9ex}\rule[0.0ex]{0.4ex}{1.3ex}\hspace{0.5ex}}}
\def\gminus{{\color{lmugreen}\rule[0.45ex]{1.4ex}{0.4ex}}}


\def\blau#1{{\color{lmugreen2}#1}}
\def\rot#1{{\color{red}#1}}
\def\gruen#1{{\color{blue}#1}}
%\def\gruen#1{{\color{gray}#1}}

%%%%%%%%%%%%%%%%%%%%%%%%%%%%%%%%%%%%%%%%%
%% Definitions & shortcuts for thesis  %%
%%%%%%%%%%%%%%%%%%%%%%%%%%%%%%%%%%%%%%%%%

\def\pdc{prior-data conflict}

\newcommand{\reals}{\mathbb{R}}

\newcommand{\dd}{\,\mathrm{d}}

\newcommand{\mbf}[1]{\mathbf{#1}}
\def\bs#1{\boldsymbol{#1}}

\newcommand{\X}{\mbf{X}}
\newcommand{\x}{\mbf{x}}

\newcommand{\uz}{^{(0)}} % upper zero
\newcommand{\un}{^{(n)}} % upper n

\def\yz{y\uz}
\def\yn{y\un}

\def\yzl{\ul{y}\uz}
\def\yzu{\ol{y}\uz}

\def\ynl{\ul{y}\un}
\def\ynu{\ol{y}\un}

\def\nz{n\uz}
\def\no{n\uo}
\def\nn{n\un}

\def\nzl{\ul{n}\uz}
\def\nzu{\ol{n}\uz}

\def\nnl{\ul{n}\un}
\def\nnu{\ol{n}\un}

\def\taux{\tau(x)}
\def\ttau{\tilde{\tau}}
\def\ttaux{\ttau(x)}

\def\Y{\mathcal{Y}}
\def\YZ{\Y\uz}
\def\YN{\Y\un}
\def\N{\mathcal{N}}
\def\NZ{\N\uz}
\def\NN{\N\un}
\def\PZ{\text{I}\!\Pi\uz}
%\def\PZero{\PZ}
\def\PN{\text{I}\!\Pi\un}
\def\MZ{\mathcal{M}\uz}
\def\MN{\mathcal{M}\un}

\newcommand{\p}{\operatorname{P}}
\newcommand{\E}{\operatorname{E}}
\newcommand{\V}{\operatorname{Var}}
\newcommand{\med}{\operatorname{med}} % Median
\newcommand{\mode}{\operatorname{mode}} % Mode
\newcommand{\logit}{\operatorname{logit}} % Mode

\def\El{\ul{\E}}
\def\Eu{\ol{\E}}

\def\Pl{\ul{\p}}
\def\Pu{\ol{\p}}

\newcommand{\mult}{\operatorname{M}}    % Multinomial Distribution
\newcommand{\norm}{\operatorname{N}}    % Normal Distribution
\newcommand{\ber}{\operatorname{Ber}}   % Bernoulli Distribution
\newcommand{\bin}{\operatorname{Binom}} % Binomial Distribution
\newcommand{\be}{\operatorname{Beta}}   % Beta Distribution
\newcommand{\B}{\operatorname{B}}   % Beta Function


\def\yzr{\rot{\yz}}
\def\ynr{\rot{\yn}}
\def\byzr{\rot{\byz}}
\def\bynr{\rot{\byn}}
\def\yzor{\rot{y\uz_1}}
\def\yzjr{\rot{y\uz_j}}
\def\yzkr{\rot{y\uz_k}}
\def\yzlr{\rot{\yzl}}
\def\yzur{\rot{\yzu}}
\def\ynjr{\rot{y\un_j}}
\def\ynlr{\rot{\ynl}}
\def\ynur{\rot{\ynu}}
\def\yzjlr#1{\rot{\ul{y}\uz_#1}}
\def\yzjur#1{\rot{\ol{y}\uz_#1}}


\def\nzg{\gruen{\nz}}
\def\nng{\gruen{\nn}}
\def\nzlg{\gruen{\nzl}}
\def\nzug{\gruen{\nzu}}
\def\nnlg{\gruen{\nnl}}
\def\nnug{\gruen{\nnu}}

\def\psib{\blau{\psi}}
\def\bpsib{\blau{{b}(\psi)}}


% ------------ shading start
\newsavebox{\tempbox}
\newcommand\leftrightshading[3]{%
  \begin{tikzfadingfrompicture}[name=inputtext]
    \node [text=white] {#1};
  \end{tikzfadingfrompicture}
  \begin{lrbox}{\tempbox}%
    \begin{tikzpicture}
      \node [text=white,inner sep=0pt,outer sep=0pt] (textnode) {#1};
      \shade[path fading=inputtext,fit fading=false,left color=#2,right color=#3]
      (textnode.south west) rectangle (textnode.north east);
    \end{tikzpicture}%
  \end{lrbox}
  % Now we use the fading in another picture:
  \usebox\tempbox{}%
}
% ------------ shading end


%\def\PZc{\mathrm I\!\Pi\uz}
\def\PZc{\leftrightshading{$\mathrm I\!\Pi\uz$}{blue}{red}}
%\def\PNc{\PN}
\def\PNc{\leftrightshading{$\mathrm I\!\Pi\un$}{blue}{red}}


\title[Bayesian Inference with Sets of Conjugate Priors]
{Bayesian Inference with Sets of Conjugate Priors: Parameter Set Shapes and Model Behaviour}

\author%[]
       {Gero Walter}

\institute{Department of Statistics\\
           Ludwig-Maximilians-Universit\"at M\"unchen (LMU)\\
                      {}%\{gero.walter; thomas\}@stat.uni-muenchen.de
}

\date{March 6th, 2013}


\titlegraphic{
%\begin{center}
\includegraphics[scale=0.032]{lmu_logos/lmu_massiv.png}
\raisebox{1.375cm}{
\includegraphics[scale=0.4]{lmu_logos/lmu_statistic.pdf}
}
%\end{center}
}

\begin{document}


\frame{
\titlepage
}

\frame{\frametitle{Outline}
\begin{enumerate}
\item Common-cause failure modelling\\ (joint work with Matthias Troffaes and Dana Kelly)
\item Generalised Bayesian inference with sets of priors\\ (joint work with Thomas Augustin)
\item Prior-data conflict and Strong prior-data agreement\\ (joint work with Thomas Augustin and Frank Coolen)
\end{enumerate}
}


\section{Common-cause Failure Modelling}


\frame{
\frametitle{Common-Cause Failures}

%\begin{center}
%\includegraphics[scale=0.4]{./graph/800px-Fukushima_I_by_Digital_Globe.jpg}\\%[-1ex]
%{\tiny Source: Wikimedia Commons, \url{http://commons.wikimedia.org/wiki/File:Fukushima_I_by_Digital_Globe.jpg}}
%\end{center}
%\vspace*{-3ex}
\begin{tikzpicture}
\uncover<1->{
\node {\includegraphics[scale=0.4]{./graph/800px-Fukushima_I_by_Digital_Globe.jpg}};
\node at (0,-3.5) {\tiny Source: Wikimedia Commons, \url{http://commons.wikimedia.org/wiki/File:Fukushima_I_by_Digital_Globe.jpg}};
}
\uncover<2>{
\draw[white,fill=white] (-0.2,-3.3) rectangle (6,0.75);
\node at (3,-1) {\includegraphics[scale=0.3]{./graph/plant-systems-diesel-generator.jpg}};
\node at (3.05,-3.05) {\parbox{6.2cm}{\tiny Source: \url{http://www.diakont.com/solutions/nuclear-energy/plant-systems/diesel-generator-control-systems/}}};
}
\end{tikzpicture}
}

\frame{
\frametitle{Common-Cause Failures}

\begin{itemize}
\item All 12 generators (for 6 reactors) at Fukushima Daiichi\\ were not available due to flooding of machine rooms\\
(Tsunami caused by T\={o}hoku earthquake)
% \begin{center}
 %\begin{minipage}{0.8\textwidth}
 \begin{alertblock}{common-cause failure} 
 \emph{simultaneous failure of several redundant components\\ due to a common or shared root cause}
 (H{\o}yland \& Rausand, 1994)
 \end{alertblock}
 %\end{minipage}
 %\end{center}
\item Reliability of redundant systems (emergency diesel generators)
\item Usually 2 -- 4 generators per reactor
\item Sufficient cooling of core if one generator works
%\item Reliability through redundancy is jeopardised by common-cause failure:%\\
\item Redundant components may not fail independently: common-cause failure
\item Must include common-cause failures in overall system reliability analysis
\end{itemize}
}


\frame{
\frametitle{Common-Cause Failure Modelling}

\begin{columns}%[T]
\begin{column}{0.7\textwidth}\centering
\includegraphics[scale=1.2]{./graph/800px-Three_Mile_Island_nuclear_power_plant.jpg}
\end{column}
\begin{column}{0.35\textwidth}\centering
\includegraphics[scale=0.95]{./graph/Graphic_TMI-2_Core_End-State_Configuration.png}
\end{column}
\end{columns}
\vspace*{1ex}
\begin{tiny}
Above: CDC, \url{http://phil.cdc.gov/phil/} ID 1194\\[1ex]
Right: Wikimedia Commons,\\[-1.8ex]
\url{http://commons.wikimedia.org/wiki/File:Graphic_TMI-2_Core_End-State_Configuration.png}
\end{tiny}
}

\subsection{Alpha-Factor Model}

\begin{frame}{Alpha-Factor Model: Definition}
  \begin{block}{Alpha-Factor Model} %\cite{1988:mosleh::common:cause}]
    Multinomial distribution $\mult(\vec{n}\mid\vec{\alpha})$ for common-cause failures \\
    in a $k$-component system\vspace*{-2ex}
    %the number of redundant components in the common-cause component group
    \begin{align*}
      p(\vec{n}\mid\vec{\alpha})=\prod_{j=1}^k\alpha_j^{n_j}\\[-5ex]
    \end{align*}
    where
    \begin{itemize}
    \item \alert{alpha-factor}
      $\alpha_j\coloneqq$
      \parbox[t]{0.6\textwidth}{%
        probability of $j$ of the $k$ components \\
        failing due to a common cause \\
        given that failure occurs
      }
    \item \alert{failure count}
      $n_j\coloneqq$ corresponding number of failures observed
    \item $\vec{n}$ denotes $(n_1,\dots,n_k)$ and $\vec{\alpha}$ denotes $(\alpha_1,\dots,\alpha_k)$
    \end{itemize}
  \end{block}
  (the model actually serves to estimate failure \emph{rates},\\ but the above is all what matters in this talk)
\end{frame}

\iffalse
\begin{frame}{Alpha-Factor Model: Inference ***raus??***}
  \begin{block}{Inference}
    involves rational functions \\
    of probabilities $\vec{\alpha}$ of common-cause failures\\
    and of total failure rate $q_t$ for individual components
  \end{block}
  \begin{center}
    \textbf{this talk: focus on $\vec{\alpha}$ only}
  \end{center}
\end{frame}
\fi

\begin{frame}{Alpha-Factor Model: Parameter Estimation}
  \begin{exampleblock}{The Good News}
    attractive feature of this model: \\
    $\vec{\alpha}$ can be estimated directly from data, e.g.\ MLE:
    \begin{align*}
      \alpha_j &= \frac{n_j}{n},\quad \text{where $\textstyle\sum_{j=1}^n n_j = n$}
    \end{align*}
  \end{exampleblock}
  \begin{alertblock}{The Bad News}
    \begin{itemize}
    \item typically, for $j\ge 2$, the $n_j$ are very low \\
      with zero being quite common for larger $j$
    \item zero counts = flat likelihoods \\
      standard techniques such as MLE can struggle \\
      to produce sensible inferences for this problem
    \end{itemize}
  \end{alertblock}
  \vspace*{-1ex}
  \begin{center}
    \textbf{\then\ need to rely on \alert{epistemic information}}
  \end{center}
\end{frame}

\subsection{Bayesian Analysis}

\begin{frame}{Bayesian Analysis: Dirichlet Prior}
  $\vec{\alpha}$ considered as uncertain parameter on which we put\dots
  \begin{block}{Dirichlet Distribution ($\to$ Dirichlet-Multinomial Model)}
  \begin{columns}
  \begin{column}{0.65\textwidth}
    \begin{align*}
      p(\vec{\alpha}\mid \nzg,\byzr) \propto \prod_{j=1}^k\alpha_j^{\nzg\yzjr-1}
    \end{align*}
  \end{column}  
  \begin{column}{0.35\textwidth}
    \rule{0ex}{3ex}where $(\nzg, \byzr)$\\ are \emph{hyperparameters}
 \end{column}
 \end{columns}\vspace*{-1.8ex}
    \begin{align*}
      \nzg &> 0\\[-1ex]
      \byzr &\in \Delta=\Big\{
        (\yzor,\dots,\yzkr)\colon \yzor\ge0,\dots,\yzkr\ge 0,\,\sum_{j=1}^k \yzjr=1
      \Big\}
    \end{align*}
  \end{block}
  \vspace*{-0.5ex}
  \begin{block}{Interpretation}
    \begin{itemize}
    \item $\byzr$ = \alert{prior expectation of $\vec{\alpha}$}, i.e., a prior guess for $\frac{n_j}{n}$, $j=1,\ldots,n$
    \item $\nzg$ = determines \alert{spread} and \alert{learning speed} (see next slide)
    \end{itemize}
  \end{block}
\end{frame}

\begin{frame}{Bayesian Analysis: Dirichlet Posterior}
  \begin{itemize}
  \item posterior density for $\vec{\alpha}$ is again Dirichlet \then\ \alert{conjugacy}\\
  \alert{update parameters}: $\nzg \to\ \nng$, $\byzr \to \bynr$
    \begin{align*}
      p(\vec{\alpha}\mid\nzg,\byzr,\vec{n})
      &=
      p(\vec{\alpha}\mid\nng,\bynr)
      \propto
%      \prod_{j=1}^k\alpha_j^{\nzg\yzjr+n_j-1}
      \prod_{j=1}^k\alpha_j^{\nng\ynjr-1}
    \end{align*}
\item posterior expectation of $\alpha_j$:
    \begin{align*}
      %\label{eq:dirichlet:posterior:predictive}
      %\colorbox{red!10!white}{$\displaystyle
      \E[\alpha_j\mid\nzg,\byzr,\vec{n}]
      &=
      \E[\alpha_j\mid\nng,\bynr] 
      =
      \int_{\Delta}\alpha_j p(\vec{\alpha}\mid\nzg,\byzr,\vec{n})\dd\vec{\alpha} \\
      &= \ynjr
      =
      %\frac{n_j+st_j}{N+s}
      %=
      %\textcolor{green!50!black}{\frac{N}{N+s}}
      %\textcolor{blue}{\frac{n_j}{N}}
      %+
      %\textcolor{green!50!black}{\frac{s}{N+s}}
      %\textcolor{blue}{t_j}
      \colorbox{lmugreen!10!white}{$\displaystyle
      \frac{\nzg}{\nzg+n} \cdot \yzjr + \frac{n}{\nzg+n} \cdot \frac{n_j}{n}
      $}
    \end{align*}
%    where $n=\sum_{j=1}^k n_j$ is total number of observations
  \end{itemize}
  %\vspace{1em}
  \begin{center}
    \textbf{we will focus on $\E[\alpha_j\mid\nng,\bynr]$} \\
    (in a decision context, this expectation would typically end up \\
    in expressions for expected utility)
  \end{center}
\end{frame}

%\subsection{Example}

\begin{frame}{Example: Prior and Data}
  (taken from Kelly \& Atwood, 2011) %\cite{2011:kelly:atwood})

  \begin{example}
    Consider a system with four redundant components ($k=4$). \\
    %The probability of
    %$j$ out of $k$ common-cause failures, given that failure has happend,
    %was denoted by $\theta_j$.
    The analyst specifies
    the following prior expectation $\mu_{\text{spec},j}$
    for each $\alpha_j$:
    \begin{align*}
      %\label{eq:example:muspec}
      \mu_{\text{spec},1}&=0.950
      &
      \mu_{\text{spec},2}&=0.030
      &
      \mu_{\text{spec},3}&=0.015
      &
      \mu_{\text{spec},4}&=0.005
    \end{align*}
    We have 36 observations, in which 35 showed one component failing,
    and 1 showed two components failing:
    \begin{align*}
      n_1&=35
      &
      n_2&=1
      &
      n_3&=0
      &
      n_4&=0
    \end{align*}
  \end{example}
\end{frame}

\begin{frame}{Example: Non-Informative Priors}
  \begin{itemize}
  \item with constrained maximum entropy prior\\ (Atwood, 1996; Kelly \& Atwood, 2011): %\cite{1996:atwood,2011:kelly:atwood}):
    \begin{align*}
      \E[\alpha_1\mid\nng,\bynr]&=0.967
      &
      \E[\alpha_2\mid\nng,\bynr]&=0.028
      \\
      \E[\alpha_3\mid\nng,\bynr]&=0.003
      &
      \E[\alpha_4\mid\nng,\bynr]&=0.001
    \end{align*}
  \item 
  with uniform prior $\yzjr=0.25$ and $\nzg=4$:
  \begin{align*}
    \E[\alpha_1\mid\nng,\bynr]&=0.9
    &
    \E[\alpha_2\mid\nng,\bynr]&=0.05
    \\
    \E[\alpha_3\mid\nng,\bynr]&=0.025
    &
    \E[\alpha_4\mid\nng,\bynr]&=0.025
  \end{align*}
  \item
  with Jeffrey's prior $\yzjr=0.25$ and $\nzg=2$:
  \begin{align*}
    \E[\alpha_1\mid\nng,\bynr]&=0.9342
    &
    \E[\alpha_2\mid\nng,\bynr]&=0.0395
    \\
    \E[\alpha_3\mid\nng,\bynr]&=0.0132
    &
    \E[\alpha_4\mid\nng,\bynr]&=0.0132
  \end{align*}
  \end{itemize}
  \alert{large variation in posterior under different non-informative priors}
  % The degree of variation in the posterior under different priors
  % is evidently somewhat alarming.
  % In the next section, we aim to robustify the model
  % by using sets of priors from the start.
\end{frame}

\subsection{Imprecise Dirichlet Model}

\begin{frame}{Imprecise Dirichlet Model: Definition}
Troffaes, Walter \& Kelly (2012): model vague prior info more cautiously
\begin{block}{Imprecise Dirichlet Model (IDM) for Common-Cause Failure}
use a \alert{set of hyperparameters}
    % \cite[p.~224, \S 5.4.3]{1991:walley} \cite[p.~32, \S 6]{1996:walley::idm},
    %\cite{1991:walley,1996:walley::idm}
    (Walley 1991, 1996):
    \begin{align*}
      %\label{eq:hyperparams:boxmodel}
      %\mathcal{H}
      \PZc
      =
      \left\{
        (\nzg,\byzr)
        \colon
        \nzg\in[\nzlg,\nzug],\,
        \byzr\in\Delta,\,
        \yzjr\in[\yzjlr{j},\yzjur{j}]
      \right\}
    \end{align*}
\end{block}\vspace*{-0.5ex}
\begin{block}{Interpretation}
\begin{itemize}
  \item we are doing a \alert{sensitivity analysis} (\'a la robust Bayes)\\ over $(\nzg,\byzr) \in \PZc$
  \item we take a \alert{set of priors} %$\text{conv}\big(\big\{p(\alpha\mid\nzg,\byzr) \colon (\nzg,\byzr) \in \PZc\big\}\big)$
  based on $\PZc$ \\
  as model for prior information (details later)
%  (resp.\ all convex mixtures of them)
\end{itemize}
\end{block}
Analyst has to specify \\
    bounds $[\nzlg,\nzug]$ and
    bounds $[\yzjlr{j},\yzjur{j}]$ for each $j\in\{1,\dots,k\}$ %\\
\end{frame}

\begin{frame}{Imprecise Dirichlet Model: Elicitation}
  \begin{itemize}
  \item   $[\yzjlr{j},\yzjur{j}]$: Cautious interpretation of prior specifications $\mu_{\text{spec},j}$:
  \begin{align*}
    [\yzjlr{1},\yzjur{1}]&=[0.950,1]
    &
    [\yzjlr{2},\yzjur{2}]&=[0,0.030]
    \\
    [\yzjlr{3},\yzjur{3}]&=[0,0.015]
    &
    [\yzjlr{4},\yzjur{4}]&=[0,0.005]
  \end{align*}
  \item  $[\nzlg,\nzug]$: Good (1965): \\ %\cite{1965:good}: \\
    \begin{center}
    reason about posterior expectations for hypothetical data
    \end{center}
  \end{itemize}
  \begin{alertblock}{}
    $\nzug$ = number of one-component failures required \\
    to reduce the upper probabilities of multi-component failure by half
  \end{alertblock}
  \begin{alertblock}{}
    $\nzlg$ = number of multi-component failures required \\
    to reduce the lower probability of one-component failure by half
  \end{alertblock}
\end{frame}

\begin{frame}{Imprecise Dirichlet Model: Elicitation}
\begin{alertblock}{}
    $\nzug$ = number of one-component failures required \\
    to reduce the upper probabilities of multi-component failure by half
  \end{alertblock}
  \begin{alertblock}{}
    $\nzlg$ = number of multi-component failures required \\
    to reduce the lower probability of one-component failure by half
  \end{alertblock}
  Reasonable values in example:
  \begin{itemize}
  \item
    $\nzlg=1$: %\\
    immediate multi-component failure
    \\
    \then\ keen to reduce lower probability for one-component failure
  \item 
    $\nzug=10$: %\\
    after observing $10$ one-component failures \\
    \then\ halve upper probabilities of multi-component failures
  \end{itemize}
  \alert{Difference between $\nzlg$ and $\nzug$} reflects a \alert{level of caution}:\\
  The rate at which we reduce upper probabilities \\
  is less than the rate at which we reduce lower probabilities \\
  %\then\ reflects a \alert{level of caution}
\end{frame}

\begin{frame}{Imprecise Dirichlet Model: Inference}
%  \begin{center}
%    prior bounds + likelihood $\to$ posterior bounds
%  \end{center}
%\vspace*{-2ex}
\begin{block}{prior bounds + likelihood $\to$ posterior bounds}
  \begin{center}
  \begin{tabular}{c|cc||cc}
    \multicolumn{1}{c}{}
  & \multicolumn{2}{c}{with $\yzjr=\mu_{\text{spec},j}$:}
  & \multicolumn{2}{c}{with bounds as earlier:} \\
    $j$ & $\El[\alpha_j\mid\PZc,\vec{n}]$ &$\Eu[\alpha_j\mid\PZc,\vec{n}]$
        & $\El[\alpha_j\mid\PZc,\vec{n}]$ &$\Eu[\alpha_j\mid\PZc,\vec{n}]$ \\
    \hline
    1 & 0.967\phantom{00} & 0.972\phantom{00} & 0.967\phantom{0} & 0.978\phantom{00}\\
    2 & 0.0278\phantom{0} & 0.0283\phantom{0} & 0.0270           & 0.0283\phantom{0}\\
    3 & 0.00041           & 0.00326           & 0\phantom{0.000}  & 0.00326\\
    4 & 0.00014           & 0.00109           & 0\phantom{0.000}  & 0.00109
  \end{tabular}
  \end{center}
\end{block} 
  \begin{itemize}
  \item \alert{Bounds}, rather than precise values, are desirable \\
    due to inferences being strongly sensitive to the prior \\
    particularly when faced with zero counts
  \item Simple ways to elicit the parameters of the model \\
    by \alert{reasoning on hypothetical data} \\
%    rather than by maximum entropy arguments
  \item Is it possible to generalise this method to other problems?
  \end{itemize}
\end{frame}

\section{Bayesian Inference with Sets of Priors}

\subsection{Canonical Conjugate Priors}

\frame{\frametitle{Canonical Conjugate Priors}

The multinomial is an example for a \alert{canonical exponential family} %(Bernardo \& Smith, 2000)
%Weighted average structure is underneath all common conjugate priors
%for exponential family sampling distributions!\\[2ex]

\begin{block}%
{$(x_1, \ldots, x_n) = \x \stackrel{iid}{\sim}$ canonical exponential family} %(Bernardo and Smith, 1994)
%, i.e.
\begin{align*}
p(\x \mid \theta) &\propto \exp\big\{\langle \psib, \tau(\x) \rangle - n \bpsib \big\}
\qquad \Big[ \psib \text{ transformation of } \theta \Big]
\end{align*}
(includes Binomial, Multinomial, Normal, Poisson, Exponential, \ldots )%\\[2ex]
\end{block}

\begin{align*}
&\text{\play\ conjugate prior:} &
p(\psib\mid\nzg,\yzr) \hspace*{2ex}
&\propto \exp\big\{ \nzg \big[\langle \psib, \yzr \rangle - \bpsib\big]\big\} \\
&\text{\play\ (conjugate) posterior:} &
p(\psib\mid\nzg,\yzr,\vec{x})
%p(\psib\mid\nng,\ynr) \hspace*{2ex}
&\propto \exp\big\{ \nng \big[\langle \psib, \ynr \rangle - \bpsib\big]\big\} %\quad \mbox{where}
\end{align*}
\begin{align*}
\mbox{where}\quad \ynr &= \frac{\nzg}{\nzg + n} \cdot \yzr + \frac{n}{\nzg + n} \cdot \frac{\tau(\x)}{n}
& &\mbox{and} & \nng &= \nzg + n
\end{align*}

}

\frame{\frametitle{Canonical Conjugate Priors}

\vspace*{-2ex}
\begin{align*}
&\text{\play\ (conjugate) posterior:} &
%p(\psib\mid\nzg,\yzr,\vec{x})
p(\psib\mid\nng,\ynr) %\hspace*{2ex}
&\propto \exp\big\{ \nng \big[\langle \psib, \ynr \rangle - \bpsib\big]\big\} %\quad \mbox{where}
\end{align*}
\vspace*{-2ex}
\begin{align*}
\mbox{where}\quad \ynr &= \frac{\nzg}{\nzg + n} \cdot \yzr + \frac{n}{\nzg + n} \cdot \frac{\tau(\x)}{n}
& &\mbox{and} & \nng &= \nzg + n
\end{align*}

\begin{block}{Interpretation}
  \begin{itemize}
  \item $\nzg$ = determines \alert{spread} and \alert{learning speed}
  \item $\byzr$ = \alert{prior expectation} of $\tau(\x)/n$ %$\frac{\tau(\vec{x})}{n}$
  \end{itemize}
\end{block}

\begin{block}{Example: Scaled Normal Data}
\begin{tabular}{r|lcl}
Data :           & $\x\mid\mu$        & $\sim$ & $\norm(\mu,1)$\\[0.5ex] %\sigma_0^2)$ \quad ($\sigma_0^2$ known)\\[0.5ex]
conjugate prior: & $\mu\mid\nzg,\yzr$ & $\sim$ & $\norm(\yzr, 1/\nzg)$  \\[0.5ex]
\cline{1-4}
posterior:       & $\mu\mid\nng,\ynr$ & $\sim$ & $\norm(\ynr, 1/\nng)$ \quad ($\frac{\tau(\x)}{n} = \bar{\x}$)\rule{0ex}{2.5ex} %& \\[0.5ex]
\end{tabular}
\end{block}
}

\subsection{Sets of Priors}

\frame{\frametitle{Bayesian Inference with Sets of Priors}

\begin{block}{Standard Bayesian inference procedure}
\centerline{prior + likelihood = posterior}
using Bayes' Rule\\
All inferences are based on the posterior\\
(e.g., point estimate = $\E[\psib\mid\nng,\ynr]$)
\end{block}

\centerline{Let hyperparameters $(\nzg, \yzr)$ vary in a set $\PZc$ \then\ \alert{set of priors}}

\begin{block}{Generalised Bayesian inference procedure}
\centerline{set of priors + likelihood = set of posteriors}
using \emph{Generalised Bayes' Rule} (GBR, Walley 1991)\\
= element-wise application of Bayes' Rule \emph{\small (set of priors must be convex)}\\
All inferences are based on the set of posteriors%\\
%(e.g., point estimate = $\big[\El[\psib\mid\nng,\ynr],\Eu[\psib\mid\nng,\ynr]\big]$)
\end{block}

}

\frame{\frametitle{Set of Priors must be Convex}

\begin{block}{Set of Priors}
%\vspace*{-1ex}
%\begin{align*}
%\MZ &= \text{conv}\big(\big\{p(\psib\mid\nzg,\byzr) \colon (\nzg,\byzr) \in \PZc\big\}\big)\vspace*{-0.5ex}%
%\end{align*}
\centerline{$\MZ = \text{conv}\big(\big\{p(\psib\mid\nzg,\byzr) \colon (\nzg,\byzr) \in \PZc\big\}\big)$}
%Model for prior information =
$\MZ$ = {\small finite convex mixtures of canonical conjugate priors defined by} $\PZc$\\
Convex set of probability distributions ensures \emph{coherence}\\ %{\small(Walley 1991)}\\
%of inference procedure
(as in standard Bayesian inference procedure)
\end{block}

Updating \& mixture commute \then\ set of posteriors can be written as\ldots

\begin{block}{Set of Posteriors}
%\vspace*{-1ex}
%\begin{align*}
%\MN &= \text{conv}\big(\big\{p(\psib\mid\nng,\bynr) \colon (\nng,\bynr) \in \PNc\big\}\big)\vspace*{-0.5ex}%
%\end{align*}
\centerline{$\MN = \text{conv}\big(\big\{p(\psib\mid\nng,\bynr) \colon (\nng,\bynr) \in \PNc\big\}\big)$}
where $\PNc = \big\{(\nng,\bynr) \colon (\nzg,\byzr) \in \PZc \big\}$.\\
%Set of posteriors = 
$\MN$ = finite convex mixtures of canonical conjugate posteriors\\ defined by set of updated hyperparameters $\PNc$
\end{block}

}

\frame{\frametitle{Generalised Bayesian Inference Procedure}

\vspace*{-1ex}
\begin{align*}
\text{single prior $(\nzg, \yzr)$}
 &\quad\mbox{\then }\quad 
\text{set of priors $\MZ$ (defined via $\PZc$)} \\
\E[\psib\mid\nzg,\yzr,\x]
 &\quad\mbox{\then }\quad 
\big[\El[\psib\mid\PZc,\x],\Eu[\psib\mid\PZc,\x]\big] \\
P(\psib\in A\mid\nzg,\yzr,\x)
 &\quad\mbox{\then }\quad 
\big[\Pl[\psib\in A\mid\PZc,\x],\Pu[\psib\in A\mid\PZc,\x]\big]
%\E[\theta]  &\quad\mbox{\then }\quad \big[\LE[\theta],\, \UE[\theta]\big] \\
%= \left[ \min_{p \in {\cal M}_\theta} \E_{p}[\theta],\,
%         \max_{p \in {\cal M}_\theta} \E_{p}[\theta]\right]\\
%P(\theta\! \in\! A) &\quad\mbox{\then }\quad \big[\ul{P}(\theta\! \in\! A),\, \ol{P}(\theta\! \in\! A)\big]
%= \big[\! \min P_{\!\!p}(\theta\! \in\! A),\,
%          \max P_{\!\!p}(\theta\! \in\! A)\big]
\end{align*}
\begin{center}
Lower/upper posterior expectation by min/max over set of posteriors %$\PZc$
\end{center}
\begin{block}{Interpretation}
\centerline{Shorter intervals \thenthen\ more precise probability statements}
\vspace*{1ex}
\parbox[t]{0.45\textwidth}{\centering \textbf{Lottery A}\\
                          Number of winning tickets:\\
                          exactly known as 5 out of 100\\
                          \then\ $P(\text{win}) = 5/100$}
\qquad
\parbox[t]{0.45\textwidth}{\centering \textbf{Lottery B}\\
                          Number of winning tickets:\\
                          not exactly known, supposedly\\
                          between 1 and 7 out of 100\\
                          \then\ $P(\text{win}) = [1/100,\, 7/100]$}
\end{block}

}

\frame{\frametitle{Generalised Bayesian Inference Procedure}

\centerline{\textbf{Shorter intervals \thenthen\ more precise probability statements}}
\begin{itemize}
\item larger $\nzg$ values as compared to $n$ \then\ larger $\PNc$\\ \quad\then\ more vague inferences\\
(more weight on imprecise prior $\MZ$ leads to more imprecise posterior $\MN$)
\item larger $n$ as compared to (range of) $\nzg$ \then\ smaller $\PNc$\\ \quad\then\ more precise inferences
\item $n \to \infty$ \then\ $\ynr$ values in $\PNc \to \frac{\tau(\x)}{n}$ \then\ `Bayesian consistency'
\item larger range of $\yzr$ in $\PZc$ \then\ larger range of $\ynr$ in $\PNc$\\ \quad  \then\ more vague inferences\\
(more imprecise prior $\MZ$ leads to more imprecise $\MN$)
\end{itemize}

}

\frame{\frametitle{Generalised Bayesian Inference Procedure}

%\begin{block}{Overview}
\begin{itemize}
\item Hyperparameter set $\PZc$ defines set of priors $\MZ$
\item Hyperparameter set $\PNc$ defines set of posteriors $\MN$
\item $\PZc \to \PNc$ is easy: $\nng = \nzg + n$, $\ynr = \frac{\nzg}{\nzg + n}\yzr + \frac{n}{\nzg + n}\frac{\tau(\x)}{n}$
%\frac{\nzg\yzr + \tau(\x)}{\nzg + n}%\dots$)
\item Quantities linear in $p(\psib\mid\nng,\bynr)$
(e.g., $\E[g(\psib)\mid\nng,\bynr]$):\\
\then\ bounds attained at ``pure'' posteriors $p(\psib\mid\nng,\bynr)$\\
\then\ \alert{straighforward to calculate:} optimise over $\PNc$ only
\item Often, optimising over $(\nng,\bynr) \in \PNc$ is also easy:\\
posterior `guess' for $\frac{\tau(\x)}{n}$ (think: $\bar{\x}$) = $\ynr$\\
\then\ \alert{closed form solution} given $\PNc$ has `nice' shape
\end{itemize}
%\end{block}

}

\subsection{Parameter Set Shapes}

\frame{\frametitle{Parameter Set Shapes}

%graphs line, rectangle, eggplant
\vspace*{-7ex}
\begin{tikzpicture}
\uncover<1>{\node {\includegraphics[scale=0.75]{../R/shape0.pdf}};
            \draw[-stealth,very thick] (-1,-0.4) -- (2.3,1.1) node [above,midway,sloped] {+ data};}
\uncover<2>{\node {\includegraphics[scale=0.75]{../R/shape1.pdf}};
            \draw[-stealth,very thick] (-1,-0.4) -- (2.3,1.1) node [above,midway,sloped] {+ data};}
\uncover<3>{\node {\includegraphics[scale=0.75]{../R/shape2.pdf}};
            \draw[-stealth,very thick] ( 0,-0.5) -- (2.3,1.1) node [above,midway,sloped] {+ data};}
\uncover<4>{\node {\includegraphics[scale=0.75]{../R/shape3.pdf}};
            \draw[-stealth,very thick] ( 0,-0.5) -- (2.3,1.1) node [above,midway,sloped] {+ data};}
\end{tikzpicture}

}

\frame{\frametitle{Parameter Set Shapes}

\begin{itemize}
\item Shape of $\PZc$ influences shape of $\PNc$
\item Shape of $\PNc$ influences model behaviour\\
\then\ shape of $\PZc$ is a crucial modelling choice
\item $\PZc = [\nzlg, \nzug] \times [\yzlr, \yzur]$ (\emph{rectangle}) %(as before)
is very easy to elicit\\
and gives good model behaviour for \alert{\pdc}
%\item \alert{\pdc}\ = ****
\end{itemize}

\begin{block}{Prior-Data Conflict}
\begin{itemize}
\item \emph{informative prior beliefs} and \emph{trusted data}\\ %\rule{0ex}{3ex}\\
(sampling model correct, no outliers, etc.) are in conflict%\\%[2ex]
\item ``[\ldots] the prior [places] its mass primarily on distributions
in the sampling model for which the observed data is surprising''\\
(Evans \& Moshonov, 2006)
\item there are not enough data to overrule the prior
\end{itemize}
\end{block}

}


\section{Prior-Data Conflict}

\subsection{Example}

\frame{\frametitle{Prior-Data Conflict: Example}

%Example Beta-Binom, show point, line, rectangle
\begin{itemize}[<+->]
\item Bernoulli observations: 0/1 observations (team wins no/yes)
\item given: a set of observations (team won 12 out of 16 matches)
\item additional to observations, we have strong prior information\\ (we are convinced that $P(\text{win})$ should be around $0.75$)
\item we are, e.g., interested in (predictive) probability $P$ that team wins in the next match
%(or some other inferences) %(predictive probability!)
%\item standard statistical model for this situation: Beta-Bernoulli/Binomial Model
%\item \textbf{\pdc:} if $P(\text{heads})$ for the coin is actually very different from our prior guess
%(i.e., prior information and data are in conflict), this should show up in the %predictive
%inferences (probability $\mbox{P}$ and, e.g., confidence intervals)
\end{itemize}
\uncover<5->{
\begin{block}{Beta-Binomial Model}
\begin{tabular}{r|lcl}
Data :           & $s\mid p, n$       & $\sim$ & $\bin(p,n)$   \\[0.5ex]
conjugate prior: & $p\mid \nzg, \yzr$ & $\sim$ & $\be(\nzg,\, \yzr)$  \\[0.5ex] %\qquad \ \ $\gruen{n\uz} = {\textstyle\sum} \alpha_i$\\[0.5ex]
\cline{1-4}
posterior:       & $p\mid \nng, \ynr$ & $\sim$ & $\be(\nng,\, \ynr)$ \quad ($\frac{\tau(\x)}{n} = \frac{s}{n}$)\rule{0ex}{2.5ex}
\end{tabular}
\end{block}
where $s$ = number of wins in the $n$ matches observed
}

}

\frame{\frametitle{Beta-Binomial Model}

\uncover<1->{
\begin{block}{Beta-Binomial Model}
\begin{tabular}{r|lcl}
Data :           & $s\mid p, n$       & $\sim$ & $\bin(p,n)$   \\[0.5ex]
conjugate prior: & $p\mid \nzg, \yzr$ & $\sim$ & $\be(\nzg,\, \yzr)$  \\[0.5ex] %\qquad \ \ $\gruen{n\uz} = {\textstyle\sum} \alpha_i$\\[0.5ex]
\cline{1-4}
posterior:       & $p\mid \nng, \ynr$ & $\sim$ & $\be(\nng,\, \ynr)$ \quad ($\frac{\tau(\x)}{n} = \frac{s}{n}$)\rule{0ex}{2.5ex}
\end{tabular}
\end{block}
}
\begin{align*}
\uncover<2->{%
P = \E[p\mid\nng,\ynr]
}%
\uncover<3->{%
= \rot{\yn}
= \frac{\gruen{\nz}}{\gruen{\nz} + n} \cdot \rot{\yz} + \frac{n}{\gruen{\nz} + n} \cdot \frac{s}{n}
}%
\end{align*}\vspace*{-3ex}
\begin{align*}
\uncover<4->{%
\gruen{\nn} &= \gruen{\nz} + n
}%
\uncover<5->{%
& \V(p\mid\nng,\ynr) &= \frac{\rot{\yn}(1-\rot{\yn})}{\gruen{\nn} + 1}
}
\end{align*}

}

\frame{\frametitle{Beta-Binomial Model}

\hspace*{-12ex}
\begin{columns}%[T]
\begin{column}{0.58\textwidth}
\begin{tikzpicture}
\pgftransformscale{0.025}
\uncover<1>{
\input{precise01.tex}
}
\uncover<2>{
\input{precise02.tex}
\draw[-stealth,very thick] (110,175) -- (220,175) node [above,midway] {12 out of 16};
}
\uncover<3->{
\input{precise03.tex}
\draw[-stealth,very thick,gray] (110,175) -- (220,175) node [above,midway,gray] {12 out of 16};
}
\uncover<4->{
\draw[-stealth,very thick] (110,102) -- (220,167) node [above,midway,sloped] {16 out of 16};
}
\end{tikzpicture}
\end{column}
\begin{column}{0.52\textwidth}
\uncover<1->{%
\begin{block}{no conflict:}
prior $\nzg = 8$, $\yzr = 0.75$\\
data $s/n = 12/16 = 0.75$
\end{block}
} %
\uncover<2->{%
\vspace*{-1.5ex}\centerline{\color{lmugreen} $\blacktriangledown$}\vspace*{-1.5ex}
\begin{block}{}
$\nng = 24$, $\ynr = 0.75$
\end{block}
} %
\uncover<4->{%
\vspace*{-1.5ex}\centerline{\color{lmugreen} $\blacktriangle$}\vspace*{-1.5ex}
}
\uncover<3->{%
\begin{block}{\pdc:}
prior $\nzg = 8$, $\yzr = 0.25$\\
data $s/n = 16/16 = 1$
\end{block}
} %
\uncover<0>{%
\vspace*{-1.5ex}\centerline{\color{lmugreen} $\blacktriangledown$}\vspace*{-1.5ex}
\begin{block}{}
$\nng \in [20, 24]$, $\ynr \in [0.73, 0.86]$
\end{block}
} %
%\uncover<5->{ %
%\then\ same predictive prob.\ P!
%}
\end{column}
\end{columns}

}

\subsection{Parameter Set Shapes}

\frame{\frametitle{Imprecise BBM with $\nzg$ fixed = IDM (Walley 1996)}

%gleich mit $\nz$ variabel?
\hspace*{-12ex}
\begin{columns}%[T]
\begin{column}{0.58\textwidth}
\begin{tikzpicture}
\pgftransformscale{0.025}
\uncover<1>{
\input{line01.tex}
}
\uncover<2>{
\input{line02.tex}
\draw[-stealth,very thick] (110,175) -- (220,175) node [above,midway] {12 out of 16};
}
\uncover<3->{
\input{line03.tex}
\draw[-stealth,very thick,gray] (110,175) -- (220,175) node [above,midway,gray] {12 out of 16};
}
\uncover<4->{
\draw[-stealth,very thick] (110,102) -- (220,167) node [above,midway,sloped] {16 out of 16};
}
\end{tikzpicture}
\end{column}
\begin{column}{0.52\textwidth}
\uncover<1->{%
\begin{block}{no conflict:}
prior $\nzg = 8$, $\yzr \in [0.7, 0.8]$\\
data $s/n = 12/16 = 0.75$
\end{block}
} %
\uncover<2->{%
\vspace*{-1.5ex}\centerline{\color{lmugreen} $\blacktriangledown$}\vspace*{-1.5ex}
\begin{block}{}
$\nng = 24$, $\ynr \in [0.73, 0.77]$
\end{block}
} %
\uncover<4->{%
\vspace*{-1.5ex}\centerline{\color{lmugreen} $\blacktriangle$}\vspace*{-1.5ex}
}
\uncover<3->{%
\begin{block}{\pdc:}
prior $\nzg = 8$, $\yzr \in [0.2, 0.3]$\\
data $s/n = 16/16 = 1$
\end{block}
} %
\uncover<0>{%
\vspace*{-1.5ex}\centerline{\color{lmugreen} $\blacktriangledown$}\vspace*{-1.5ex}
\begin{block}{}
$\nng \in [20, 24]$, $\ynr \in [0.73, 0.86]$
\end{block}
} %
%\uncover<5->{ %
%\then\ same imprecise predictive prob.\ $[\Pl,\Pu]$!
%}
\end{column}
\end{columns}

}


\frame{\frametitle{Imprecise BBM with $[\nzlg,\nzug]$ (Walley 1991, \S 5.4.3)}

\hspace*{-12ex}
\begin{columns}%[T]
\begin{column}{0.58\textwidth}
\begin{tikzpicture}
\pgftransformscale{0.025}
\uncover<1>{
\input{rect01.tex}
}
\uncover<2>{
\input{rect02.tex}
\draw[-stealth,very thick] (110,175) -- (193,175) node [above,midway] {12 out of 16};
}
\uncover<3>{
\input{rect03.tex}
\draw[-stealth,very thick,gray] (110,175) -- (193,175) node [above,midway,gray] {12 out of 16};
}
\uncover<4>{
\input{rect04.tex}
\draw[-stealth,very thick,gray] (110,175) -- (193,175) node [above,midway,gray] {12 out of 16};
\draw[-stealth,very thick] (110,102) -- (213,180) node [above,midway,sloped] {16 out of 16};
%\draw[-stealth,very thick, postaction={decorate,decoration={raise=2pt,text along path, %
%                                       text={16 out of 16}}}] (110,95) .. controls +(right:30) and +(down:30) .. (210,183);
%\draw[-stealth,very thick] (110,95) .. controls +(right:10) and +(down:10) .. (210,183) node [below,midway,sloped] {16 out of 16};
}
\end{tikzpicture}
\end{column}
\begin{column}{0.52\textwidth}
\uncover<1->{%
\begin{block}{no conflict:}
prior $\nzg \in [4, 8]$, $\yzr \in [0.7, 0.8]$\\
data $s/n = 12/16 = 0.75$
\end{block}
} %
\uncover<2->{%
\vspace*{-1.5ex}\centerline{\color{lmugreen} $\blacktriangledown$}\vspace*{-1.5ex}
\begin{block}{}
%$\gruen{\nn} \in [20, 24]$,
\centering $\ynr \in [0.73, 0.77]$
%\centering ``spotlight'' shape
\end{block}
} %
\uncover<3->{%
\begin{block}{\pdc:}
prior $\nzg \in [4, 8]$, $\yzr \in [0.2, 0.3]$\\
data $s/n = 16/16 = 1$
\end{block}
} %
\uncover<4->{%
\vspace*{-1.5ex}\centerline{\color{lmugreen} $\blacktriangledown$}\vspace*{-1.5ex}
\begin{block}{}
%$\gruen{\nn} \in [20, 24]$,
\centering $\ynr \in [0.73, 0.86]$
%\centering ``banana'' shape
\end{block}
} %
\end{column}
\end{columns}

}


\frame{\frametitle{Parameter Set Shapes: Discussion}

\begin{itemize}
\item $\PZc = \nzg \times [\yzlr, \yzur]$:\\% \thenthen\
IDM (Walley 1996), Quaghebeur \& de Cooman (2005)
 \begin{itemize}
 \item posterior parameter set has same form $\PNc = \nng \times [\ynlr, \ynur]$ %simpel model description,
 \item optimise over $[\ynlr, \ynur]$ only %(often, posteriors are stoch.\ ordered along $\ynr$)
 \item no \pdc\ reaction: same imprecision as without conflict (just like precise priors)
 \end{itemize}
\item $\PZc = [\nzlg,\nzug] \times [\yzlr, \yzur]$: %\thenthen\
Walley (1991, \S 5.4.3),\\ \emph{generalized iLUCK-models} (Walter \& Augustin 2009) 
 \begin{itemize}
 \item still simple to elicit, allows flexible weighing of prior and data
 \item additional imprecision in case of \pdc\\ \then\ more cautious inferences
 \item $\PNc$ have non-trivial forms (banana / spotlight) %posterior parameter sets 
 \item however, closed form for $\min / \max \ynr$ over $\PNc$
 \item general optimisation over $\PNc$ more difficult, but doable
 \item \textbf{R} package \texttt{luck}: do optimisation over $\PZc$ actually 
 \end{itemize}
\end{itemize}

}


\frame{\frametitle{Parameter Set Shapes: Discussion}

\begin{itemize}
\item Need a range of $\nzg$ values for \pdc\ reaction
\item Other set shapes are possible, but may be more difficult to elicit
\item Prior information may be such that range of $\yzr$ changes with $\nzg$ (or vice versa)
\item \emph{Near-ignorance priors}: $\PZc$ such that prior inferences are \emph{vacuous},
but posterior inferences are informative
 \begin{itemize}
 \item IDM (Walley 1996): range of $\yzjr = (0,1)\ \forall j$
 \item Benavoli \& Zaffalon (2012): range of $\yzr = (-\infty,+\infty)$\\
 while $\nzug$ decreasing with $\yzr$\\ %$\nzg$ bounded above
 (to avoid $\nzg |\yzr| = \infty$, i.e.\ vacuous posterior inferences)
 \end{itemize}
\end{itemize}

}

\frame{\frametitle{Parameter Set Shapes: Outlook}

Work in progress (joint work with Frank Coolen):\\
parameter set shape enabling\ldots
\begin{itemize}
\item additional imprecision in case of \pdc\ (as before)
\item less imprecision for \emph{strong prior-data agreement}
\end{itemize}
%This work uses 
via a different parametrisation of priors suggested by Mik Bickis % (in preparation)

}

\subsection{Conclusion}

\frame{\frametitle{Conclusion}

\begin{itemize}
\item Conjugate priors are a convenient tool for Bayesian inference\\ but have some pitfalls
 \begin{itemize}
 \item Hyperparameters are easy to interpret and elicit
 \item Averaging property makes calculations simple,\\
 but inadequate model behaviour in case of \pdc
 \end{itemize}
\item Sets of conjugate priors maintain advantages \& mitigate pitfalls %find a sweet spot in between
 \begin{itemize}
 \item Hyperparameter set shape is important %a crucial modeling choice
 \item Reasonable choice: \emph{rectangular} $\PZc = [\nzlg, \nzug] \times [\yzlr, \yzur]$
 \item Bounds for hyperparameters easy to interpret and elicit
 \item Additional imprecison in case of \pdc\\
 leads to cautious inferences if, and only if, caution is needed
 \item Shape for less imprecision for strong prior-data agreement is in development
 \end{itemize}
\end{itemize}

}


\frame{\frametitle{References}

\nocite{Walter2009a-short,1994:hoyland,1965:good,1991:walley,1996:walley::idm,2011:kelly:atwood,1996:atwood,Troffaes2013a,2006:evans,2005:quaeghebeurcooman-short}

\printbibliography[heading=none]

%\bibliographystyle{plainnat}
%\bibliography{../bib/eigene,../bib/itip-refs,../bib/other-refs}
}

\end{document}