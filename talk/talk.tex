\documentclass{beamer}

\usetheme{LMU}

\usepackage[utf8]{inputenc}
\usepackage[T1]{fontenc}

\usepackage[german]{babel}

\usepackage{amssymb, amsmath, amsfonts, enumerate}
\usepackage{bm}
%\usepackage{dsfont}
\usepackage{pxfonts}
\usepackage{xcolor}

\usepackage{tikz}
\usetikzlibrary{%
   arrows,%
   calc,%
   fit,%
   patterns,%
   plotmarks,%
   shapes.geometric,%
   shapes.misc,%
   shapes.symbols,%
   shapes.arrows,%
   shapes.callouts,%
   shapes.multipart,%
   shapes.gates.logic.US,%
   shapes.gates.logic.IEC,%
   er,%
   automata,%
   backgrounds,%
   chains,%
   topaths,%
   trees,%
   petri,%
   mindmap,%
   matrix,%
   calendar,%
   folding,%
   fadings,%
   through,%
   positioning,%
   scopes,%
   decorations.fractals,%
   decorations.shapes,%
   decorations.text,%
   decorations.pathmorphing,%
   decorations.pathreplacing,%
   decorations.footprints,%
   decorations.markings,%
   shadows}
\usepackage{hyperref}

\setbeamertemplate{blocks}[rounded][shadow=true]
\definecolor{lmugreen2}{RGB}{0,120,94}
%\definecolor{lmugreen2}{RGB}{0,148,64}
\definecolor{lmugreen}{RGB}{0,140,84}
\definecolor{unidurham}{RGB}{126,49,123}

\parindent0pt
\setlength{\unitlength}{1ex}
\setlength{\fboxsep}{0ex}

\def\then{{\color{lmugreen}$\rule[0.35ex]{2ex}{0.5ex}\!\!\!\blacktriangleright$}}
%\def\then{{\color{lmugreen}$\blacktriangleright\!\blacktriangleright$}}
%\def\then{{\color{lmugreen}$\blacktriangleright$}}
%\def\then{{\color{lmugreen}$\Rrightarrow$}}
%\def\then{{\color{lmugreen}$\rhd$}}
%\def\then{{\color{lmugreen}$\gg\!\!\!\!\!\gg$}}
%\def\then{{\color{lmugreen}${\mathbf{\gg}}$}}

\def\rthen{{\color{lmugreen}$\rule[0.35ex]{0.5ex}{0.95ex}\rule[0.35ex]{1.3ex}{0.5ex}\!\!\!\blacktriangleright$}}

\def\thenthen{{\color{lmugreen}$\blacktriangleleft\!\!\!\rule[0.35ex]{2ex}{0.5ex}\!\!\!\blacktriangleright$}}

\def\gplus{{\color{lmugreen}\rule[0.45ex]{1.4ex}{0.4ex}\hspace{-0.9ex}\rule[0.0ex]{0.4ex}{1.3ex}\hspace{0.5ex}}}
\def\gminus{{\color{lmugreen}\rule[0.45ex]{1.4ex}{0.4ex}}}


\def\blau#1{{\color{lmugreen2}#1}}
\def\rot#1{{\color{red}#1}}
\def\gruen#1{{\color{blue}#1}}
%\def\gruen#1{{\color{gray}#1}}

%%%%%%%%%%%%%%%%%%%%%%%%%%%%%%%%%%%%%%%%%
%% Definitions & shortcuts for thesis  %%
%%%%%%%%%%%%%%%%%%%%%%%%%%%%%%%%%%%%%%%%%

\def\pdc{prior-data conflict}

\newcommand{\reals}{\mathbb{R}}

\newcommand{\dd}{\,\mathrm{d}}

\newcommand{\mbf}[1]{\mathbf{#1}}
\def\bs#1{\boldsymbol{#1}}

\newcommand{\X}{\mbf{X}}
\newcommand{\x}{\mbf{x}}

\newcommand{\uz}{^{(0)}} % upper zero
\newcommand{\un}{^{(n)}} % upper n

\def\yz{y\uz}
\def\yn{y\un}

\def\yzl{\ul{y}\uz}
\def\yzu{\ol{y}\uz}

\def\ynl{\ul{y}\un}
\def\ynu{\ol{y}\un}

\def\nz{n\uz}
\def\no{n\uo}
\def\nn{n\un}

\def\nzl{\ul{n}\uz}
\def\nzu{\ol{n}\uz}

\def\nnl{\ul{n}\un}
\def\nnu{\ol{n}\un}

\def\taux{\tau(x)}
\def\ttau{\tilde{\tau}}
\def\ttaux{\ttau(x)}

\def\Y{\mathcal{Y}}
\def\YZ{\Y\uz}
\def\YN{\Y\un}
\def\N{\mathcal{N}}
\def\NZ{\N\uz}
\def\NN{\N\un}
\def\PZ{\text{I}\!\Pi\uz}
%\def\PZero{\PZ}
\def\PN{\text{I}\!\Pi\un}
\def\MZ{\mathcal{M}\uz}
\def\MN{\mathcal{M}\un}

\newcommand{\p}{\operatorname{P}}
\newcommand{\E}{\operatorname{E}}
\newcommand{\V}{\operatorname{Var}}
\newcommand{\med}{\operatorname{med}} % Median
\newcommand{\mode}{\operatorname{mode}} % Mode
\newcommand{\logit}{\operatorname{logit}} % Mode

\def\El{\ul{\E}}
\def\Eu{\ol{\E}}

\def\Pl{\ul{\p}}
\def\Pu{\ol{\p}}

\newcommand{\mult}{\operatorname{M}}    % Multinomial Distribution
\newcommand{\norm}{\operatorname{N}}    % Normal Distribution
\newcommand{\ber}{\operatorname{Ber}}   % Bernoulli Distribution
\newcommand{\bin}{\operatorname{Binom}} % Binomial Distribution
\newcommand{\be}{\operatorname{Beta}}   % Beta Distribution
\newcommand{\B}{\operatorname{B}}   % Beta Function


\def\yzr{\rot{\yz}}
\def\ynr{\rot{\yn}}
\def\yzlr{\rot{\yzl}}
\def\yzur{\rot{\yzu}}
\def\ynlr{\rot{\ynl}}
\def\ynur{\rot{\ynu}}

\def\nzg{\gruen{\nz}}
\def\nng{\gruen{\nn}}
\def\nzlg{\gruen{\nzl}}
\def\nzug{\gruen{\nzu}}
\def\nnlg{\gruen{\nnl}}
\def\nnug{\gruen{\nnu}}

\title[Bayesian Inference with Sets of Conjugate Priors]
{Bayesian Inference with Sets of Conjugate Priors: Parameter Set Shapes and Model Behaviour}

\author%[]
       {Gero Walter}

\institute{Department of Statistics\\
           Ludwig-Maximilians-Universit\"at M\"unchen (LMU)\\
                      {}%\{gero.walter; thomas\}@stat.uni-muenchen.de
}

\date{March 6th, 2013}


\titlegraphic{
%\begin{center}
\includegraphics[scale=0.032]{lmu_logos/lmu_massiv.png}
\raisebox{1.375cm}{
\includegraphics[scale=0.4]{lmu_logos/lmu_statistic.pdf}
}
%\end{center}
}

\begin{document}


\frame{
\titlepage
}

\section{Common-cause Failure Modelling}


\frame{
\frametitle{Common-Cause Failures}

\begin{center}
\includegraphics[scale=0.4]{./graph/800px-Fukushima_I_by_Digital_Globe.jpg}\\%[-1ex]
{\tiny Source: Wikimedia Commons, \url{http://commons.wikimedia.org/wiki/File:Fukushima_I_by_Digital_Globe.jpg}}
\end{center}
}


\frame{
\frametitle{Common-Cause Failures}

\begin{itemize}
\item Reliability of redundant systems (emergency diesel generators)
\item Usually 2 -- 4 generators per reactor
\item Sufficient cooling of core if one generator works
%\item Reliability through redundancy is jeopardised by common-cause failure:%\\
\item redundant component may not fail independently:
 \begin{alertblock}{common-cause failure} 
 \emph{simultaneous failure of several redundant components\\ due to a common or shared root cause}
 \end{alertblock}
\item All 12 generators (for 6 reactors) at Fukushima Daiichi not available due to flooding of machine rooms\\
(Tsunami from T\={o}hoku earthquake)
\item Include common cause failures in overall system reliability analysis
\end{itemize}
}


\frame{
\frametitle{Common-Cause Failure Modelling}

\begin{columns}%[T]
\begin{column}{0.7\textwidth}\centering
\includegraphics[scale=1.2]{./graph/800px-Three_Mile_Island_nuclear_power_plant.jpg}\\%[-1ex]
{\tiny Source: CDC, \url{http://phil.cdc.gov/phil/} ID 1194}
\end{column}
\begin{column}{0.3\textwidth}\centering
\includegraphics[scale=0.8]{./graph/Graphic_TMI-2_Core_End-State_Configuration.png}\\%[-1ex]
{\tiny Source: Wikimedia Commons, \url{http://commons.wikimedia.org/wiki/File:Graphic_TMI-2_Core_End-State_Configuration.png}}
\end{column}
\end{columns}
}

\subsection{Alpha-Factor Model}

\begin{frame}{Alpha-Factor Model: Definition}
  \begin{definition}[Alpha-Factor Model] %\cite{1988:mosleh::common:cause}]
    multinomial distribution for common-cause failures \\
    in a $k$-component system\vspace*{-1ex}
    %the number of redundant components in the common-cause component group
    \begin{align*}
      p(\vec{n}\mid\vec{\alpha})=\prod_{j=1}^k\alpha^{n_j}
    \end{align*}
    where
    \begin{itemize}
    \item \alert{alpha-factor}
      $\alpha_j\coloneqq$
      \parbox[t]{0.6\textwidth}{%
        probability of $j$ of the $k$ components \\
        failing due to a common cause \\
        given that failure occurs
      }
    \item \alert{failure count}
      $n_j\coloneqq$ corresponding number of failures observed
    \item $\vec{n}$ denotes $(n_1,\dots,n_k)$ and $\vec{\alpha}$ denotes $(\alpha_1,\dots,\alpha_k)$
    \end{itemize}
  \end{definition}
  (there's much more to this, but the above is all what matters in this talk)
\end{frame}


\end{document}